\documentclass{beamer}

\usepackage[utf8]{inputenc}


%Information to be included in the title page:
\title{Projet de recherche}
\subtitle{Résumé de l'article \textbf{Handling Algebraic effects}}
\author{Jordan Ischard}
\institute{Université d'Orléans}
\date{2020-2021}



\begin{document}



\AtBeginSection[]
{
  \begin{frame}
    \frametitle{Sommaire}
    \tableofcontents[currentsection,subsectionstyle=show/shaded/hide]
  \end{frame}
}

\AtBeginSubsection
{
  \begin{frame} 
	\frametitle{Sommaire}
	\tableofcontents[sectionstyle=show/shaded/hide,subsectionstyle=show/shaded/hide]
  \end{frame}
}



\frame{\titlepage}

\section{Introduction}

\subsection{Problème de l'impératif dans un langage fonctionnel}
\begin{frame}
\frametitle{Ajout de principes impératifs dans un langage fonctionnel}

	\begin{block}{But}
		Ajouter des fonctionnalités en plus dans les langages fonctionnels
	\end{block}

	\begin{alertblock}{Problème}
		Ajoute des effets de bords que les langages fonctionnels purs non pas.
	\end{alertblock}

	\begin{exampleblock}{Exemple}
		On peut avoir des effets de bords pour les appels mémoires ou encore
		les entrées/sorties.
	\end{exampleblock}

	\begin{block}{Idée}
		Créer une structure qui va gérer ces effets.
	\end{block}
\end{frame}

\subsection{Proposition pour gérer les effets}
\begin{frame}
	\frametitle{Réponses déjà proposées}

	\begin{block}{Les Monades}
		Eugenio Moggi a proposé le principe de \alert{Monade} pour gérer les effets.
	\end{block}

	\begin{exampleblock}{Les Monades en Haskell}
		coucou
	\end{exampleblock}

	\begin{block}{Les théories d'équations}
		Plotkin et Power ont proposé des opérations comme source des effets et une
		\alert{théorie d'équation} pour décrire leurs propriétés.
	\end{block}

	\begin{block}{L'article}
		On reprend le principe de Plotkin et Power et on crée des gestionnaires 
		pour ces effets. 
	\end{block}
\end{frame}

\section{Gestion des effets dans l'article}

\subsection{Intuition et principe}
\begin{frame}
	\frametitle{Comment gérer les effets ?}
	TODO : opérations comme source des effets etc.
\end{frame}

\subsection{Mise en place dans une syntaxe}
\begin{frame}
	\frametitle{Qu'ajoute-on pour mettre en place la gestion des effets ?}
	TODO : syntaxe, sémantique et fonctionnement
\end{frame}

\subsection{Exemple}
\begin{frame}
	\frametitle{dérogation à la lecture seule}
	TODO : l'exemple en gros
\end{frame}

\section{Validé des gestionnaires}

\subsection{Gestionnaire générique indécidable}
\begin{frame}
	\frametitle{Les gestionnaires en général sont indécidable}
	TODO : Trop complexe, obliger de décider entre laisser l'utilisateur faire tout
	en partant du principe qu'il peut se louper ou le contraindre
\end{frame}

\subsection{Sous-ensemble décidable}
\begin{frame}
	\frametitle{Gestionnaire décidable sous certaines conditions}
	TODO : principe de gestionnaire simple, de famille uniformément simple etc.
\end{frame}

\section{Confrontation}

\subsection{Les divergences}
\begin{frame}
	\frametitle{Motivation et fonctionnement divergent}
	TODO : motivation, source des effets etc.
\end{frame}

\subsection{Exemple de conversion entre langage}
\begin{frame}
	\frametitle{Passage du langage de l'article vers \textbf{erpl}}
	TODO : l'exemple en gros
\end{frame}

\subsection{Difficulté d'implémentation}
\begin{frame}
	\frametitle{Une histoire d'appels systèmes}
	TODO : expliquer le problème de l'appel système et du gestionnaire global implicite
\end{frame}

\section{Conclusion}

\begin{frame}
	\frametitle{Sample frame title}
	
	In this slide, some important text will be
	\alert{highlighted} because it's important.
	Please, don't abuse it.
	
	\begin{block}{Remark}
	Sample text
	\end{block}
	
	\begin{alertblock}{Important theorem}
	Sample text in red box
	\end{alertblock}
	
	\begin{examples}
	Sample text in green box. The title of the block is "Examples".
	\end{examples}
\end{frame}


\end{document}