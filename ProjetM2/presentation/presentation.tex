\documentclass{beamer}

\usepackage[utf8]{inputenc}
\usepackage{listings}
\usepackage{xcolor}

\definecolor{codegreen}{rgb}{0,0.6,0}
\definecolor{codegray}{rgb}{0.5,0.5,0.5}
\definecolor{codepurple}{rgb}{0.58,0,0.82}
\definecolor{backcolour}{rgb}{0.95,0.95,0.92}

\lstdefinestyle{mystyle}{
    backgroundcolor=\color{backcolour},   
    commentstyle=\color{codegreen},
    keywordstyle=\color{orange},
    numberstyle=\tiny\color{codegray},
    stringstyle=\color{codepurple},
    basicstyle=\ttfamily\footnotesize,
    breakatwhitespace=false,         
    breaklines=true,                 
    captionpos=b,                    
    keepspaces=true,                 
    numbers=left,                    
    numbersep=5pt,                  
    showspaces=false,                
    showstringspaces=false,
    showtabs=false,                  
    tabsize=2
}

\lstset{style=mystyle}


%Information to be included in the title page:
\title{Projet de recherche}
\subtitle{Résumé de l'article \textbf{Handling Algebraic effects}}
\author{Jordan Ischard}
\institute{Université d'Orléans}
\date{2020-2021}



\begin{document}



\AtBeginSection
{
  \begin{frame}
    \frametitle{Sommaire}
    \tableofcontents[currentsection,subsectionstyle=show/shaded/hide]
  \end{frame}
}

\AtBeginSubsection
{
  \begin{frame} 
	\frametitle{Sommaire}
	\tableofcontents[subsectionstyle=show/shaded/hide]
  \end{frame}
}



\frame{\titlepage}

\section{Introduction}

\subsection{Problème de l'impératif dans un langage fonctionnel}
\begin{frame}
\frametitle{Ajout de principes impératifs dans un langage fonctionnel}

	\begin{block}{But}<1->
		Ajouter des fonctionnalités en plus dans les langages fonctionnels
	\end{block}

	\begin{alertblock}{Problème}<2->
		Ajoute des effets de bords que les langages fonctionnels purs non pas.
	\end{alertblock}

	\begin{exampleblock}{Exemple}<3->
		On peut avoir des effets de bords pour les appels mémoires ou encore
		les entrées/sorties.
	\end{exampleblock}

	\begin{block}{Idée}<4->
		Créer une structure qui va gérer ces effets.
	\end{block}
\end{frame}

\subsection{Proposition pour gérer les effets}
\begin{frame}[fragile]
	\frametitle{Réponses déjà proposées}

	\begin{block}{Les Monades}<1->
		Eugenio Moggi a proposé le principe de \alert{Monade} pour gérer les effets.
	\end{block}

	\begin{exampleblock}{Les Monades en Haskell}<2->
		\begin{lstlisting}[language=Haskell]
instance Monad Maybe where
	return x = Just x
			
	Nothing >>= f = Nothing
	Just x >>= f  = f x
					
	fail _ = Nothing
		\end{lstlisting}
	\end{exampleblock}

	\begin{block}{Les théories d'équations}<3->
		Plotkin et Power ont proposé des opérations comme source des effets et une
		\alert{théorie d'équation} pour décrire leurs propriétés.
	\end{block}
\end{frame}

\section{Gestion des effets dans l'article}

\subsection{Intuition et principe}
\begin{frame}
	\frametitle{Comment gérer les effets ?}
	\begin{block}{Intuition}<1->
		Reprendre le principe de Plotkin et Power et l'associer aux travaux de Benton et Kennedy sur
		les gestionnaires.
	\end{block}

	\begin{block}{Principe}<2->
		Les effets vont avoir pour source des \alert{opérations} et vont être géré par 
		un \alert{tableaux associatifs (map)} contenu dans un gestionnaire. 
	\end{block}
\end{frame}

\subsection{Mise en place dans une syntaxe}
\begin{frame}[fragile]
	\frametitle{Qu'ajoute-on pour mettre en place la gestion des effets ?}
	\begin{block}{Syntaxe ajouté}<1->
		On part de la syntaxe proposé par Levy dans son article sur \textit{call-by-push-value}.
		\begin{align*}
			source~de~l'effet~:~&\textbf{op}_V(x:\beta.M')\\
			structure~de~gestion~:~&M~\textbf{handled~with}~H~to~x:A.N\\
			gestionnaire~:~&\{\textbf{op}_{z:\alpha}(k:\beta \rightarrow \underline{C}) \mapsto M_{\textbf{op}}\}_{\textbf{op} : \alpha \rightarrow \beta}
		\end{align*}
	\end{block}

	\begin{block}{Fonctionnement}<2->
		On reprend la strucutre de gestion ci-dessus 
		avec $\textbf{op}_V(y.M') \in M$ et $\{\textbf{op}_z(k) \mapsto M_ {\textbf{op}}\} \in H$.
			\alert{\[M_{\textbf{op}}[V/z,M'[W/y]~\textbf{handled~with}~H~\textbf{to}~x:A.N/k(W)]\]}
	\end{block}
\end{frame}

\subsection{Exemple}
\begin{frame}
	\frametitle{dérogation à la lecture seule}
	
	\begin{exampleblock}{Valeur Temporaire sans modification dans la mémoire}<1->
		Le gestionnaire définit pour l'effet est le suivant : 
		\[H_{temporary} = \{\textbf{get}_{l:\textbf{loc}}(k:\textbf{nat} \rightarrow \underline{C}) \mapsto k(n)\}\]

		Prenons l'expression suivante : 
		\begin{align*}
			& \textbf{let}~n:\textbf{nat}~\textbf{be}~20~\textbf{in}\\
			& \textbf{get}_l(x:\textbf{nat}.get_l(y:\textbf{nat}.\textbf{return}~x+y))\\
			&\textbf{handled~with}~H_{temporary}~\textbf{to}~z:A.\textbf{return}~z+2
		\end{align*}
	\end{exampleblock}

	\begin{alertblock}{La réponse}<2->
		C'est 42 ! 
	\end{alertblock}
\end{frame}

\section{Validé des gestionnaires}

\subsection{Gestionnaire générique indécidable}
\begin{frame}
	\frametitle{Les gestionnaires en général sont indécidable}
	TODO : Trop complexe, obliger de décider entre laisser l'utilisateur faire tout
	en partant du principe qu'il peut se louper ou le contraindre
\end{frame}

\subsection{Sous-ensemble décidable}
\begin{frame}
	\frametitle{Gestionnaire décidable sous certaines conditions}
	TODO : principe de gestionnaire simple, de famille uniformément simple etc.
\end{frame}

\section{Confrontation}

\subsection{Les divergences}
\begin{frame}
	\frametitle{Motivation et fonctionnement divergent}
	TODO : motivation, source des effets etc.
\end{frame}

\subsection{Exemple de conversion entre langage}
\begin{frame}
	\frametitle{Passage du langage de l'article vers \textbf{erpl}}
	TODO : l'exemple en gros
\end{frame}

\subsection{Difficulté d'implémentation}
\begin{frame}
	\frametitle{Une histoire d'appels systèmes}
	TODO : expliquer le problème de l'appel système et du gestionnaire global implicite
\end{frame}

\section{Conclusion}

\begin{frame}
	\frametitle{Sample frame title}
	
	In this slide, some important text will be
	\alert{highlighted} because it's important.
	Please, don't abuse it.
	
	\begin{block}{Remark}
	Sample text
	\end{block}
	
	\begin{alertblock}{Important theorem}
	Sample text in red box
	\end{alertblock}
	
	\begin{examples}
	Sample text in green box. The title of the block is "Examples".
	\end{examples}
\end{frame}


\end{document}