La partie interprétation sera omit dans ce résumé car il fait appel à des notions de catégorie et des notions poussées sur les modèles. Je n'ai pas la prétention d'être en capacité de le comprendre entièrement et de l'expliquer.

\subsection{La théorie des effets}

	Les propriétés des effets sont décrites avec des équation entre patrons $T$. Ils décrivent la forme général de tous les calculs sans présumer des types. $T$ est définit par :
	\[T := z(V)~|~\textbf{match}~V~\textbf{with}~\langle x,y\rangle \mapsto T~|~\textbf{match}~V~\textbf{with}~\{l(x_l) \mapsto T_l\}_{l \in L}~|~\textbf{op}_V(x:\beta.T)\]
	
	avec $z$ appartenant à un ensemble de variable de patron. Dans les patrons, on se limite à la signature des valeurs. Ceux sont des valeurs qui peuvent être typées par $\Gamma \vdash V:\alpha$, avec $\Gamma = x_1:\alpha_1,...,x_m:\alpha_m$.
	\medbreak
	
	Les patrons sont construit dans le contexte de variables de valeur $\Gamma$ et un contexte de patron : $Z = z_1:\alpha_1,...,z_n:\alpha_n$. Attention, $z_j:\alpha_j$ ne représente pas une valeur de type $\alpha_j$ mais un calcul qui dépend d'une telle valeur.
	Le jugement de typage pour un patron bien-formé est $\Gamma~|~Z \vdash T$ et est donné par les règles suivantes :
	
	\begin{align*}
		&\dfrac{\Gamma \vdash V:\alpha}{\Gamma~|~Z \vdash z(V)}~~(z:\alpha \in Z) &
		&\dfrac{\Gamma \vdash V : A \times B\quad\quad\Gamma~x:\alpha,y:\beta~|~Z \vdash T}{\Gamma~|~Z \vdash \textbf{match}~V~\textbf{with}~\langle x,y\rangle  \mapsto T} \\\\
		&\dfrac{\Gamma \vdash V : \sum_{l \in L}\alpha_l\quad\quad\Gamma~,x_l:\alpha_l~|~Z \vdash T_l~~(l\in L)}{\Gamma~|~Z \vdash \textbf{match}~V~\textbf{with}~\{ l(x_l) \mapsto T_l\}_{l \in L}} &
		&\dfrac{\Gamma \vdash V:\alpha\quad\quad\Gamma,x:\beta~|~Z \vdash T}{\Gamma~|~Z \vdash \textbf{op}_V(x:\beta.T)}~~(\textbf{op}:\alpha \rightarrow \beta)\\
	\end{align*}
	
	Une théorie d'effet $\tau$ est un ensemble fini d'équation $\Gamma~|~Z \vdash T_1 = T_2$, avec $T_1$ et $T_2$ bien formé par rapport à $\Gamma$ et $Z$.
	
	\begin{exemple}
		On reprend les symboles d'opérations de l'Exemple \ref{exStateExc}.
		
		\begin{enumerate}
			\item[] \textbf{Exceptions:} La théorie d'effet est un ensemble vide, car les exceptions ne satisfont aucune équations non-trivial.
			
			\item[] \textbf{États:} La théorie d'effet est définie par les équations suivantes (on n'écrit pas les contextes pour une meilleur lecture) :
			\begin{align*}
				\textbf{get}_l(x.z) &= z\\	
				\textbf{get}_l(x.\textbf{set}_{l,x}(z)) &= z\\	
				\textbf{set}_{l,x}(\textbf{set}_{l,x'}(z)) &= \textbf{set}_{l,x'}(z)\\
				\textbf{set}_{l,x}(\textbf{get}_{l}(x'.z(x'))) &= \textbf{set}_{l,x}(z(x))\\
				\textbf{get}_{x}(x.\textbf{get}_{l}(x'.z(x,x'))) &= \textbf{get}_{l}(x.z(x,x))\\
				\textbf{set}_{l,x}(\textbf{set}_{l',x'}(z)) &= 
				\textbf{set}_{l',x'}(\textbf{set}_{l,x}(z)) & (l \neq l')\\
				\textbf{set}_{l,x}(\textbf{get}_{l'}(x'.z(x'))) &= 
				\textbf{get}_{l'}(x'.\textbf{set}_{l,x}(z(x'))) & (l \neq l')\\
				\textbf{get}_{l}(x.\textbf{get}_{l'}(x'.z(x,x'))) &= 
				\textbf{get}_{l'}(x'.\textbf{get}_{l}(x.z(x,x'))) & (l \neq l')
			\end{align*}
			
			\item[] \textbf{Non-déterminisme:} La théorie d'effet est définie par les équations suivants :  
			\begin{align*}
				\textbf{choose}(z,z) &=~z\\
				\textbf{choose}(z_1,z_2) &=~\textbf{choose}(z_2,z_1)\\
				\textbf{choose}(z_1,\textbf{choose}(z_2,z_3)) &=~\textbf{choose}(\textbf{choose}(z_1,z_2),z_3)
			\end{align*}
		
			\item[] \textbf{Temps:} La théorie d'effet est définie par les équations suivants :  
			\begin{align*}
				\textbf{delay}_0(z)  &=~z\\
				\textbf{delay}_{t_1}(\textbf{delay}_{t_2}(z)) &=~ \textbf{delay}_{t_1 + t_2}(z)
			\end{align*}
		
		
			\item[] \textbf{Exception destructrice:} La signature des exceptions destructives est l'union de la signature de l'état et de l'exception. La théorie d'effet comprend toute les équations de l'état plus les équations suivantes :  			
			\begin{align*}
				\textbf{get}_l(x.\textbf{raise}_e()) &=~\textbf{raise}_e()\\
				\textbf{set}_{l,x}(\textbf{raise}_{e}()) &=~\textbf{raise}_{e}()
			\end{align*}
			\bigbreak
			
			Les équations ci-dessus impliquent que les opérations sur la mémoire suivit par une levée d'exception est la même chose que juste lever une exception. Cela implique que toutes les opérations sur la mémoire sont discutables si une exception apparaît. La théorie des exceptions destructives est discutée comme celle du "rollback" dans \cite{DBLP:journals/tcs/HylandPP06}.
		\end{enumerate}
	\end{exemple}