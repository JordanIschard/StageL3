\usepackage[utf8]{inputenc}
%\usepackage[T1]{fontenc}
\usepackage{version}

\usepackage{amssymb,amsmath,amsthm}
\usepackage{graphicx}
\usepackage{enumerate}
\usepackage{tikz}
\usetikzlibrary{matrix,arrows}
\usepackage{todonotes}
\usepackage{fullpage}
\usepackage[french,linesnumbered,lined,boxed,commentsnumbered,ruled,vlined]{algorithm2e}

\renewcommand{\contentsname}{Table des matières}   

%\usepackage[ruled,vlined,linesnumbered,noresetcount]{algorithm2e}

\SetKwInput{KwIn}{Entrée}%
\SetKwInput{KwOut}{Sortie}%
%\SetKwIF{Si}{SinonSi}{Sinon}{si}{alors}{sinon si}{sinon}{fin si}%
%\SetKwFor{Tq}{tant que}{faire}{fin tq}%
%\SetKwFor{For}{Pour}

\usepackage{tikz}
\usetikzlibrary{positioning}
\usetikzlibrary{calc}

\usepackage[affil-it]{authblk}

\renewcommand*{\Authsep}{, }
\renewcommand*{\Authand}{, }
\renewcommand*{\Authands}{, }

\title{Résumé de l'article \textbf{Handling algebraic effects}}

\author[1]{Jordan Ischard}

\affil[1]{Université d'Orléans}

\date{}

% ===== ENVIRONMENTS ===== %
\newtheorem{theorem}{Théorème}
\newtheorem{lemma}{Lemme}
\newtheorem{corollary}[theorem]{Corollaire}
\newtheorem{claim}[theorem]{Claim}
\newtheorem{proposition}[theorem]{Proposition}
\newtheorem{definition}{Définition}
\theoremstyle{definition}
\newtheorem*{remark}{Remarque}
\newtheorem{exemple}{Exemple}

\newenvironment{proofclaim}{
	\noindent \emph{Proof.}
}{%
	\hfill $\diamond$ \\
}

% ===== COMMANDE POUR DEFINIR UN PROBLEME ===== %
\newcommand{\probleme}[4]{

    \vspace{0.4cm}
    \fbox{
        \begin{minipage}{0.95 \linewidth}
            \centerline{\textsc{\underline{#1}}}
            \textbf{Entr\'ee} : #2 \\ \textbf{#4} : #3
        \end{minipage}
    }
    \vspace{0.4cm}

}

\newcommand{\RMV}{\todo[inline]{Paragraphe à supprimer}}
\renewcommand*{\proofname}{Preuve}
