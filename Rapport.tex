\documentclass[10pt,a4paper]{article}
\usepackage[utf8]{inputenc}
\usepackage[a4paper,margin=1.5cm]{geometry}
\usepackage[T1]{fontenc}
\usepackage[french]{babel}
\usepackage{amssymb}
\usepackage{enumitem}
\usepackage{pifont}
\usepackage{hyperref}
\usepackage{textcomp}
\usepackage{ulem}

\begin{document}
		
	\title{\textbf{Rapport de stage \\ Développement d'un noyau de programmation synchrone}}
	\date{25 Mars 2019}
	\author{Jordan Ischard\\3ème année de licence Informatque \\ Université d'Orléans}
	\maketitle
	\newpage
	
	
	\tableofcontents
	\newpage
	
	\section{Remerciement}
		Avant tout développement sur mon sujet de stage, j'aimerai remercié mes 2 professeurs qui m'ont encadrés pour m'avoir permis de faire ce stage de recherche et de m'avoir aidé tout le long de celui-ci. J'ai beaucoup appris grâce à eux. Je remercie donc Madame Bousdira et Monsieur Dabrowski pour tout.
	\bigbreak
	
	\section{Introduction}
	
	
	
	\newpage
	
	\section{Journal de bord}
		\subsection{Mars}
			\subsubsection{Semaine 25 au 29 Mars}
			\medbreak
			Afin de me remettre dans le contexte du stage et de comprendre ces problématiques, le lundi a été entièrement consacré à la relecture et lecture des deux articles ,proposés durant l'entretien, un sur le RéactiveML\hyperref[ReactiveML]{[1]} et un sur le ZINC\hyperref[ZINC]{[2]}.
			\medbreak
			Toutes mes notes, qui sont un résumé de ce que j'ai compris, sont données dans la section \hyperref[Recherche]{\textbf{Recherche d'Informations}}.
			\medbreak
			
			Le mardi matin a servie à clarifier mes questions en une petite liste présentée lors de la réunion. La réunion s'est déroulée l'après-midi et a permis d'apporter des réponses aux questions pré-écrite le matin et à fixer des objectifs. 
			\\La réunion en a fixé 2:
			\smallbreak
			\begin{enumerate}
				\item Impl\'{e}menter les $\lambda$-calculs
				\item Impl\'{e}menter la machine SECD
			\end{enumerate}
			2 semaines m'ont été accordé pour réaliser ces objectifs dans le langage de programmation \textit{Ocaml}. J'ai eu, pour cela, un article traitant des $\lambda$-calculs\hyperref[Calculi]{[3]} afin de m'aider dans ma démarche.\\
			Il a été décidé que tous les programmes seront partagé sur un github qui a pour adresse :\\ \textit{https://github.com/JordanIschard/StageL3}.\\
			Le reste de la journée a été utilisé pour commencer ce dit article.
			\medbreak
			
			La compréhension des $\lambda$-calcul a été étalé sur le reste de la semaine.
			\\
			Cependant l'avancement a été perturbé le mercredi à cause d'un problème de compréhension des règles de priorité dans les $\lambda$-calculs. Ce temps perdu en compréhension a été compensé dans l'écriture de ce rapport.
			\medbreak
			M. Dabrowski m'a aidé à régler ce problème le jeudi matin ce qui m'a permis d'assimilé en presque totalité les $\lambda$-calculs et de commencer l'implémentation de ceux-ci en Ocaml.
			
			\subsection{Avril}
			
			\subsubsection{Semaine du 1 au 5 Avril}
			L'implémentation des $\lambda$-calculs a presque été terminé à l'exception du parser, de l'$\alpha$-réduction et d'un petit bug de renommage lié à la règle de $\beta$-réduction suivante :
			\smallbreak
			\begin{itemize}
				\item [] ($\lambda$X$_{1}$.M$_{1}$)[X$_{2}$ $\leftarrow$ M$_{2}$] = ($\lambda$X$_{3}$.M$_{1}$[X$_{1}$ $\leftarrow$ X$_{3}$][X$_{2}$ $\leftarrow$ M$_{2}$]) 
				\\où X$_{1}$$\neq$X$_{2}$, X$_{3}$$\notin$ FV(M$_{2}$) 
				et X$_{3}$$\notin$ FV(M$_{1}$)$\backslash${X$_{1}$} 
			\end{itemize}
			La réunion a changé les objectifs de base , en gardant le langage Ocaml, par les objectifs suivants : 
			\begin{enumerate}
				\item Implémenter les $\lambda$-calculs
				\item Implémenter la machine CC et par extension la machine SCC
				\item Implémenter la machine CK et par extension la machine CEK
				\item Implémenter la machine SECD
			\end{enumerate}
			Une semaine supplémentaire m'a été accordé pour faire cela.
			\medbreak
			La compréhension des différentes machines ont été faite le lundi après la réunion, le mardi et le mercredi, toutes les informations récolté sont dans la partie \hyperref[Machine]{traitant des différentes machines}. \smallbreak
			Cependant rien a été implémenté à cause d'un problème lié à Ocaml et Emacs. En effet, Ocaml propose un système de module pour pouvoir séparer des parties de codes dans différents fichiers.
			\smallbreak
			Malheureusement après une après-midi complète de recherche sur ce sujet, Je n'ai pas réussi à trouver une solution. Un mail a donc été envoyé à M. Dabrowski et Mme Bousdira pour avoir une aide le mercredi matin.
			\smallbreak 
			La structure des méthodes et leurs contenus a commencé à être écrit sur papier jusqu'au jeudi midi. En effet, M. Dabrowski m'a aidé avec les modules en Ocaml. Il fallait laisser tomber emacs et son intéraction dynamique et faire une compilation grâce à \textit{ocamlbuild}.
			\smallbreak
			L'implémentation du langage ISWIM a été fait le vendredi . La machinne CC est presque terminé mais un bug lié aux opérations créer une erreur. L'implémentation de la machine SCC sera faite durant la semaine prochaine. Le problème de l'implémentation des $\lambda$-calculs a été fixé ainsi que le bug de la machine CC durant le week-end.
			
			\newpage
		
		\subsubsection{Semaine du 8 au 12}
			L'implémentation des machines CK,CEK et SECD ont été faite le lundi et le mardi. La réunion hebdomadaire ne s'est pas déroulé le lundi par manque de temps, donc les nouvelles consignes de mes professeurs n'ont pas été donné. En attendant le mardi après-midi a été utilisé pour lire la suite de l 'article \hyperref[Calculi]{[3]}.
			\smallbreak
			Une implémentation du systèmes d'erreur a été faite le mercredi matin. 
			Une réunion a été placé le jeudi à 14h. Le temps entre mercredi matin et jeudi après-midi fut utilisé à la lecture du même article et plus précisément la mise en place de gestionnaire d'erreur et le système d'affectation.
			\smallbreak
			La réunion a permis de mettre en place un objectif ainsi qu'un objectif secondaire :
			\begin{enumerate}
				\item Implémenter une concurrence dans la machine SECD avec les notions suivantes :
					\begin{enumerate}
						\item \textbf{spawn t} : lance un nouveau thread et le mets en attente
						\item \textbf{present s in t1 t2} : si le signal s est présent prends t1 sinon attend le prochain temps logique et prends t2
						\item \textbf{emit s} : émet le signal s
						\item \textbf{signal s in t} : initialise le signal s pour t 
					\end{enumerate}
				\item Implémenter une gestion d'erreur dans la machine SECD 
			\end{enumerate}
			Ces 2 objectifs doivent être atteint en 3 semaines maximum.
			\medbreak
			Le reste de la semaine a été utilisé pour écrire les règles de la machine SECD avec les nouvelles notions. En effet, toute la difficulté du rajout de notion dans une machine est de ne pas "casser" les règles de base.Pour cela, un petit point théorique est obligatoire.
			\smallbreak
			Tout d'abord, il faut distinguer deux types de thread différentes :
			\begin{itemize}
				\item ceux qui attendent
				\item ceux qui sont bloquées
			\end{itemize}
			\bigbreak
			
			
			Ce qui oblige de rajouter à notre machine SECD deux élément en plus, $\widehat{W}$ les éléments qui attendent et $\widehat{ST}$ les éléments qui sont bloqués.
			\bigbreak
			
			
			La question suivante se pose : \textbf{Que doit contenir les éléments de ces deux listes ?}
			\smallbreak
			
			La réponse est : le nécessaire au fonctionnement des threads indépendamment les uns des autres. Comme dit plus haut ils sont indépendant les uns des autres , leurs piles, leurs environnements, leurs chaînes de contrôle et leurs dépôts sont obligatoirement indépendants pour ne pas créer de conflits mais par contre la liste d'attente et la liste d'éléments bloqués sont eux communs à tous.
			\medbreak
			
			Donc $\widehat{W}$ contient une liste de sauvegarde de la machine SECD c'est-à-dire une liste de $\widehat{D}$. $\widehat{ST}$ contient une liste de couple contenant une sauvegarde et le signal attendu.On a pour l'instant :
			
			\begin{itemize}
				\item[] $\widehat{W}$ = $\{\widehat{D},...\}$
				\item[] $\widehat{ST}$ = $\{\langle s,\widehat{D}\rangle,...\}$ avec s un signal 
			\end{itemize}
		
			Ensuite, il faut savoir quand un signal est émit et enfin le plus dur quand est-ce que le signal n'est pas émit.
			\\
			La liste de signaux émits doit-il être commun ou privé à chacun ? La question mérite d'être posé et après plusieurs essais le plus simple reste commun car pour vérifier la présence d'un signal émit par un autre thread avant il faut partager cette liste. Du coup on a $\widehat{SI}$ qui est une liste de signaux émits.
			\smallbreak
			On dit qu'un signal n'est pas émit si pendant tout un instant logique on a pas d'émission de ce signal. \textbf{Comment est représenté la fin d'un instant logique ?} Il est représenté par la $\widehat{W}$ vide, $\widehat{D}$ vide et $\widehat{C}$ vide. 
			\smallbreak
			Du coup quand ces trois conditions sont réunit, on peut prendre tous les éléments de $\widehat{ST}$ et les travailler en prenant en compte que le signal n'a pas été émit.
			\smallbreak
			Une première version sans le \textbf{signal s in t} a commencé à être conçu.
			\newpage
			
			
		\subsubsection{Semaine du 15 au 19}
			La 1ère version des règles de la nouvelle machine SECD fût terminé le lundi et vous pouvez trouvez ces dites règles en \hyperref[SECDConc1]{Annexe}. 
			Un ajout a été fait en plus dans la 1ère version, quand on tombe sur un élément comme \textbf{spawn} ou  \textbf{emit} il faut continuer de faire fonctionner la machine. Donc une constante \textbf{Remp} a été ajouté dans l'optique de ne pas casser le fonctionnement de la machine SECD de base.
			\\ Cette version est inutilisable car elle part dans l'optique que tous les signaux sont déjà initialisé.
			\smallbreak
			Pour palier à ce problème une 2ème version de ces règles fût créer le lundi matin, vous pouvez les retrouver en \hyperref[SECDConc2]{Annexe} comme pour la 1ère version.
			\medbreak
			L'implémentation de ces règles a été faite le lundi après-midi. Cette implémentation semble concluante mais il reste potentiellement des failles. Cette semaine aucune réunion sera faites donc je présenterai mes travaux à la réunion de la semaine prochaine pour avoir l'approbation de mes professeurs.
			\smallbreak
			L'objectif principal étant atteint, je me suis penché sur l'objectif secondaire le mardi. 
			\smallbreak
			Tout d'abord, j'ai rajouté un système d'erreur dans la machine SECD qui arrête le fonctionnement de celle-ci et renvoie un message d'erreur et non qui s'arrête par la gestion d'erreur de Ocaml. Le système fonctionnant, j'ai changé les erreurs par des \textbf{Throw} en prévision de son utilisation futur. M'inspirant des règles créé pout la machine CCH présenté dans \hyperref[Calculi]{[3]} j'ai aussi rajouté un autre élément à la machine SECD concurrente : le \textbf{Handler}. Le \textbf{Handler} doit contenir une sauvegarde de la machine secd au complet. Donc on a:
			
			\begin{itemize}
				\item[] $\widehat{H}$ = Vide | $\langle erreur,\langle\widehat{S},\widehat{E},\widehat{C},\widehat{W},\widehat{ST},\widehat{SI},\widehat{D},\widehat{H}\rangle\rangle$
			\end{itemize}
			\smallbreak
			
			Pour aller de pair avec le \textbf{Throw} j'ai aussi rajouté un \textbf{Catch}.
			Une 3ème version des règles de la machine SECD concurrente fût créé le mardi après-midi. Elle se trouve ,elle aussi, en \hyperref[SECDConc3]{Annexe}. L'implémentation a été faite le mercredi matin.
			\medbreak
			
			Ayant terminé le travail demandé, j'ai envoyé un mail le mercredi pour avoir plus ample information sur le travail à faire ensuite. Un mail contenant la suite m'a été envoyé le mercredi vers midi. Après avoir créer des threads et les avoir fait travaillé indépendamment, le but est ici de donner la possibilité de partager des valeurs.
			\smallbreak
			
			 Pour cela il faut :
			\begin{enumerate}
				\item Un \textbf{identifiant unique} à chaque thread
				\item une \textbf{liste de valeurs} liés à un signal et un thread précis
				\item une opération \textbf{put s v} qui met une valeur v dans la liste de valeurs associé à s qui est lui même associé au thread 
				\item une opération \textbf{get id s} qui récupère une valeur v dans la liste de valeurs associé à s qui est lui même associé au thread id 
			\end{enumerate}
			Un temps de réflexion sur ces ajouts ,et les questions qui vont avec, a été faite le mercredi après-midi. Les questions étaient lié à quand est-ce que l'on met à jour les valeurs partagés principalement mais aussi comment avec le get on récupère les informations. 
			\newpage
			Tout le jeudi et le vendredi ont été utilisé pour voir les ajouts à la machine. Voilà le fruit de mes réflexions.
			Reprenons dans l'ordre :
			\begin{itemize}
				\item Un \textbf{identifiant unique} à chaque thread : Cela implique deux choses. 
					\begin{enumerate}
						\item Il faut garder en mémoire l'id de chaque thread, en soit il faut rajouté un élément dans la machine car un thread en cours est une machine secd classique donc on rajoute $\widehat{I}$ l'identifiant et il faut le garder aussi dans la sauvegarde faite pour la mise en attente;
						\item Il faut que l'id soit unique, pour cela deux possibilités : 
							\begin{enumerate}
								\item un id d'un thread fini peut être réutilisé mais cela peut créer des conflit plus tard (De plus, le vendredi un mail de M. Dabrowski confirmera mes craintes et privilegiera la 2nd option dans un premier temps)
								\item un id d'un thread fini est bloqué jusqu'à la fin du fonctionnement de la machine. Pour cela, il faut garder un entier dans la machine qui va être pris pour chaque nouveau thread et qui va s'incrémenter. On a donc $\widehat{AI}$ un entier. 
							\end{enumerate}
					\end{enumerate}
				\item une \textbf{liste de valeurs} liés à un signal et un thread précis : Cela implique de modifier la liste des signaux pour ajouter une liste de valeur et un identifiant de thread liés.
				\smallbreak
				De plus, dans la version précdente les signaux initialisés étaient dans l'environnement mais ici cela impliquerai de garder deux listes de valeurs indépendante pour le même signal et le même thread. Ce serai un gâchis d'espace mémoire. Du coup, on va tout mettre dans $\widehat{SI}$ avec la forme suivante : \smallbreak
				 $\widehat{SI}$  = $\{ \langle id , \{\langle signal , \{valeur,...\},init,emit\rangle\}\rangle,...\}$ 
				 \\ avec id = un identifiant de thread
				 \\ avec init = un booléen indiquant si il est initialisé ou non
				 \\ avec emit = un booléen indiquant si il est émit ou non
				 \medbreak
				 
				 On peut légitimemment se poser la question de pourquoi un booléen init, et je vais vous expliquer pourquoi.
				 \smallbreak
				 En effet, quand on mets un signal dans cette liste c'est qu'il a été initialisé à un moment dans le fonctionnement de la machine. \textbf{A un moment} est la partie importante car en effet un signal peut ne plus être initialisé dans le thread en cours (par exemple en terminant un \textit{signal s in t}) mais avoir été émit avant. Donc pour gagner de l'espace je préfère un booléen à la place d'une liste de valeurs en double.
				 
				 \item les opérateurs \textbf{put} et \textbf{get} ,qui eux une fois les points précédémment expliqués seront implémentés, rajouteront deux règles mais ne provoquera pas de gros changement.
				 \smallbreak 
				 Seul le \textbf{get} va être un peu plus difficile car on a la contrainte de ne pouvoir prendre qu'une fois une valeur dans une liste. On va donc devoir rajouter dans la liste des valeurs partagées que l'on va définir plus loin une liste d'identifiant à chaque élément de la liste de valeurs pour simuler un pointeur.
			\end{itemize}
			\bigbreak
			
			
		\subsubsection{Semaine du 22 au 26}
			
			L'implémentation de la nouvelle structure a pris le mardi au complet et le mercredi matin. Mais manquait de plusieurs choses et avait des failles. 
			\smallbreak
			La liste de signal $\widehat{SI}$ était un bon commencement mais il me manquait la liste de valeurs partagées. Pour cela $\widehat{SI}$ va devenir un couple de liste de la forme $\langle\widehat{CS},\widehat{SSI}\rangle$. Les signaux courants comme 1ère élément qui a la structure de l'ancien $\widehat{SI}$ et le 2nd élément qui va regroupé les valeurs partagées. Avant de parler de sa structure, il faut remettre en question la structure de $\widehat{CS}$. 
			\smallbreak
			Après avoir fini l'implémentation et mis en phase de test la machine j'ai remarqué 2 problèmes : 
				\begin{itemize}
					\item trier par threads puis par signaux est moins pratique que trier par signaux puis par threads. Du coup la structure de $\widehat{CS}$ sera $\{ \langle signal , \{\langle id , \{valeur,...\},init\rangle\},emit\rangle,...\}$ .
					\item quand on créé un nouveau thread, j'avais omis de dupliqué les signaux initialisé plus tôt pour le nouveau thread créé dans la liste des signaux. Ceci a été fixé. 
				\end{itemize} 
			\bigbreak
			
			
			Une 4ème version a donc été écrit avec les put et get et les améliorations et peut être retrouvé dans la \hyperref[SECDConc4]{Dernière version de la machine SECD concurrente}. La définition des éléments de la machine y sont aussi. Toutes ces modifications ont été faites durant le reste de la semaine.
		\newpage
		\subsection{Mai}
		
		\subsubsection{Semaine du 29 Avril au 3}
			Ayant terminé l'implémentation de la 4ème version, les tests ont été fait le lundi en attendant la réunion hebdomadaire. La réunion a été reporté du coup par mail j'ai demandé la suite du travail à produire. Madame Bousdira m'a donc répondu et voici le nouvelle objectif :
				\begin{enumerate}
					\item Faire une preuve prouvant le déterminisme de la machine
				\end{enumerate} 
			\bigbreak
			Ne sachant pas comment mis prendre, les recherches ont été faite sur internet et j'ai trouvé cet article\hyperref[Turing]{[4]} montrant le déterminisme de la machine de Turing par un automate déterministe. Les différentes tentatives fut des échecs car il me manquait la différenciation entre le déterminisme d'un automate et d'une machine, je m'explique :
				\begin{itemize}
					\item[] \textbf{Définition du déterminisme d'un automate} : Un automate fini déterministe, parfois abrégé en AFD (en anglais deterministic finite automaton, abrégé en DFA) est un automate fini dont les transitions à partir de chaque état sont déterminées de façon unique par le symbole d'entrée. \hyperref[AFD]{[5]}
					\item[] \textbf{Définition du déterminisme d'une machine via un algorithme} : Un algorithme déterministe est un algorithme qui, étant donné une entrée particulière, produira toujours la même sortie, avec la machine sous-jacente passant toujours par la même séquence d'états. 
					\smallbreak
					Formellement, un algorithme déterministe calcule une fonction mathématique ; une fonction ayant une valeur unique pour n'importe quelle entrée dans son ensemble de définition, l'algorithme produit cette valeur en sortie. \hyperref[MFD]{[6]}
				\end{itemize} 
		\medbreak
		Ce problème de compréhension a été détecté par Mme Bousdira le jeudi matin et réglé le jeudi après-midi avec Mme Bousdira et Mr Dabrowski. Une preuve par induction a été indiqué comme une solution.\smallbreak
		Avant cela la réécriture des nouvelles règles au propre a été nécessaire et fait le vendredi pour que mes professeurs puissent vérifier mon travail avant de commencer la preuve.  
		
		\subsubsection{Semaine du 6 au 10}
		
		La réunion s'est déroulé le lundi matin. Les règles ont été vérifié par mes professeurs et certains point modifiés et ce sont les suivants :
		\begin{itemize}
			\item \textbf{signal s in C} a été changé par \textbf{init s}, c'est-à-dire que le signal n'est plus initialisé pour une chaîne de contrôle précise mais pour l'entièreté de la machine durant son fonctionnement.
			\item \textbf{get} appliquera directement l'élément pris dans la liste de valeur à l'abstraction au lieu de le faire passivement en ajoutant dans la chaîne de contrôle.
			\item \textbf{Unit} n'est pas nécessaire car quand on convertit le langage ISWIM en langage SECD on vérifie la syntaxe du coup les applications sont forcément bon syntaxiquement.
			\item \textbf{get} retournera une erreur quand il aura fini de prendre dans la liste de valeurs.
			\item \textbf{throw erreur} sera propagé plutôt que vérifié directement.
		\end{itemize}
		\medbreak
		Il a été demandé aussi de mettre un nom explicite pour chaque règles ainsi que des commentaires explicatifs. 
		\smallbreak
		Ces modifications ont été faite le lundi après-midi et mardi. Un mail a été envoyé pour avoir l'approbation des nouvelles règles avant de commencer à faire la preuve du déterminisme. Quelques améliorations ont été faites le jeudi et une relance par mail a été faite.
		
		
		\subsubsection{Semaine du 13 au 17}
		
		\subsubsection{Semaine du 20 au 24}
	\newpage
	
	\section{Recherche d'Informations}\label{Recherche}
		\subsection{Le Réactive ML}
				Modèle de programmation $\longrightarrow$ concurrence coopérative 
				\smallbreak
				L'analyse est découpé en 2 sous analyse :
				\begin{itemize}
					\item \textbf{statique} : système de type et d'effet
					\item \textbf{réactive} : détecter les erreurs de concurrence
				\end{itemize}
				\bigbreak
			
			
				Ordonnancement coopératif$\ast$ $\longrightarrow$ \textit{\small{chaque processus va r\'{e}guli\`{e}rement "laisser la main aux autres"}}
				\smallbreak
				Ordonnancement pr\'{e}emptif $\longrightarrow$ \textit{\small{le syst\`{e}me va "donner un temps de parole" \`{a} chacun}}
				\medbreak
				
				Points fort :
				\begin{itemize}
					\item \textit{\small{impl\'{e}mentation s\'{e}quentielle efficace}}
					\item \textit{\small{pas de probl\`{e}me de parall\'{e}lisme}}
				\end{itemize}
				\medbreak
				
				Points faible : 
				\begin{itemize}
					\item \textit{\small{responsabilit\'{e} de la r\'{e}activit\'{e} au dev}}
				\end{itemize}
				\medbreak
				
				Le mod\`{e}le r\'{e}actif synchrone d\'{e}finit une notion de temps logique qui est une succession d'instant.
				\smallbreak
				\textbf{
					Un programme est r\'{e}actif is son ex\'{e}cution fait progresser les instants logique.
				}
				\medbreak
				
				\textbf{Exemple}
				\medbreak
				
				\begin{enumerate}
				 	\item let process clock timer s =
				 	\item $\quad$ let time = ref(Unix.gettimeofday()) in
				 	\item $\quad$ loop
				 	\item $\quad$ $\quad$ let time' = Unix.gettimeofday() in
				 	\item $\quad$ $\quad$ if time' -. !time >= timer
				 	\item $\quad$ $\quad$ then(emit s(); time := time')
				 	\item $\quad$ end 
				\end{enumerate}
				\medbreak
				
				Le probl\`{e}me ici est que le contenu de la boucle peut s'effectuer instantann\'{e}ment or il faut attendre un instant logique pour. Du coup, on doit ajouter une \textbf{pause entre la ligne 6 et 7}.
				\bigbreak
				
				Une \textbf{condition suffisante} pour q'un \textbf{processus r\'{e}cursif soit r\'{e}actif} est qu'il ait \textbf{toujours} au moins \textbf{un instant logique entre l'instanciation du processus et l'appel r\'{e}cursif}.
				
				
		
			\newpage
			
			\subsection{Les  $\lambda$-calculs}
				
				\subsubsection{Les règles de $\beta$-réduction}
					\begin{itemize}
						\item X$_{1}$[X$_{1}$ $\leftarrow$ M] = M
						\item X$_{2}$[X$_{1}$ $\leftarrow$ M] = X$_{2}$ 
						\\où X$_{1}$$\neq$X$_{2}$
						\item ($\lambda$X$_{1}$.M$_{1}$)[X$_{1}$ $\leftarrow$ M$_{2}$] =  ($\lambda$X$_{1}$.M$_{1}$)
						\item ($\lambda$X$_{1}$.M$_{1}$)[X$_{2}$ $\leftarrow$ M$_{2}$] = ($\lambda$X$_{3}$.M$_{1}$[X$_{1}$ $\leftarrow$ X$_{3}$][X$_{2}$ $\leftarrow$ M$_{2}$]) 
						\\où X$_{1}$$\neq$X$_{2}$, X$_{3}$$\notin$ FV(M$_{2}$) 
						et X$_{3}$$\notin$ FV(M$_{1}$)$\backslash${X$_{1}$} 
						\item (M$_{1}$ M$_{2}$)[X $\leftarrow$ M$_{3}$] = 
						(M$_{1}$[X $\leftarrow$ M$_{3}$] M$_{2}$[X $\leftarrow$ M$_{3}$])
					\end{itemize}
			
				\subsubsection{Les règles de réduction générale}
					\begin{itemize}
						\item ($\lambda$X$_{1}$.M) $\alpha$ ($\lambda$X$_{1}$.M[X$_{1}$ $\leftarrow$ X$_{2}$])
						où X$_{2}$$\notin$  FV(M)
						\item (($\lambda$X$_{1}$.M$_{1}$)M$_{2}$) $\beta$ M$_{1}$[X $\leftarrow$ M$_{2}$]
						\item ($\lambda$X.(M X)) $\eta$ M 
						où X$\notin$  FV(M)
					\end{itemize}
				
				\textit{La réduction générale \textbf{n} = $\alpha \cup \beta \cup \eta$}.
				
				\subsubsection{Les règles de priorité}
					\begin{itemize}
						\item Application associative à gauche : M1 M2 M3 = ((M1 M2)M3)
						\item Application prioritaire par rapport au abstraction : $\lambda$X.M1 M2 =  $\lambda$X.(M1 M2)
						\item Les abstractions consécutives peuvent être regroupé : $\lambda$XYZ.M = ($\lambda$X.($\lambda$Y.($\lambda$Z.M)))
					\end{itemize}
				
				
				\subsubsection{Comment savoir si on a une forme normal ?}
				
				Une expression est une forme normale si on ne peut pas r\'{e}duire l'expression via une $\beta$ ou $\eta$ r\'{e}dution.
				\medbreak
				
				\textbf{Th\'{e}or\`{e}me de la forme normale} :
				Si on peut r\'{e}duire L tels que L =$_{n}$ M et L =$_{n}$ N et que N et M sont en forme normal alors M = N à n renommage pr\`{e}s.
				\medbreak	
				
				\textbf{Th\'{e}or\`{e}me de Church-Rosser ( pour =$_{n}$ )}:
				Si on a M =n N, alors il existe un L' tels que M $\twoheadrightarrow n_{n}$ L' et N $\twoheadrightarrow n_{n}$ L'.
				\medbreak
				
				Certaines lambda calcul expression n'a pas de forme normal comme: (($\lambda$x.x x) ($\lambda$x.x x). 
				\newline
				D'autres en ont une mais on peut rentrer dans une boucle infini de r\'{e}duction si on choisi la mauvaise r\'{e}duction.
				\medbreak
				Le probl\`{e}me est quand on \'{e}value un argument de la fonction qui n'est jamais utilis\'{e}. Pour palier \`{a} ça, on utilise la strat\'{e}gie d'appliquer toujours les $\beta$ et $\eta$ réduction le plus \`{a} gauche. Ces règles sont les suivantes:
				\medbreak
				
				\begin{itemize}
					\item M $\longrightarrow_{\bar{n}}$ N if M $\beta$ N
					\item M $\longrightarrow_{\bar{n}}$ N if M $\eta$ N
					\item ($\lambda$X.M) $\longrightarrow_{\bar{n}}$ ($\lambda$X.N)
					\item (M N) $\longrightarrow_{\bar{n}}$ (M' N) if M $\longrightarrow_{\bar{n}}$ M'
					\\ et $\forall$L, (M N) $\beta$ L impossible et (M N) $\eta$ L impossible
					\item (M N) $\longrightarrow_{\bar{n}}$ (M N') if N $\longrightarrow_{\bar{n}}$ N'
					\\ et M est une forme normale
					\\ et  $\forall$L, (M N) $\beta$ L impossible et (M N) $\eta$ L impossible
				\end{itemize}
				
				Cette solution est s\^{u}r mais reste peut utilis\'{e} car elle est assez lente. 
			\newpage
			\subsection{ISWIM}	
			
				ISWIM à une grammaire étendue de la grammaire des $\lambda$-calcul.
				\smallbreak
				M,N,L,K =
				\begin{itemize}
					\item[|] X (les variables)
					\item[|] ($\lambda$X.M)
					\item[|] (M M)
					\item[|] b (les constantes b )
					\item[|] (o$^{n}$ M ... M) avec o$^{n}$ les fonctions primitives
				\end{itemize}
				\medbreak
				
				V,U,W =
				\begin{itemize}
					\item[|] b
					\item[|] X
					\item[|] $(\lambda X.M)$
				\end{itemize}
				
				
				Une application n'est jamais une valeur alors qu'une abstraction est toujours une valeur.
				\medbreak
				Les règles de $\beta$-réductions sont les mêmes que celle pour les $\lambda$-calcul avec 2 ajouts:
				\begin{itemize}
					\item b[X $\longleftarrow$ M] = b
					\item (o$^{n}$ M$_{1}$ ... M$_{n}$)[X $\longleftarrow$ M] = (o$^{n}$ M$_{1}$[X $\longleftarrow$ M] ... M$_{n}$[X $\longleftarrow$ M])
				\end{itemize}
				
				La réduction est la même quand lambda calcul mais on vérifie juste que la réduction est faite avec une valeur. (($\lambda$X.M) V) $\beta_{v}$ M[X $\longleftarrow$ V]. Cette restriction permet une sorte d'ordre dans les calculs.
				\smallbreak
				$\eta$ et $\alpha$ réduction ne sont plus vue comme tel.L'$\eta$-réduction n'est pas utilisé et l'$\alpha$-réduction sera utilisé que pour chercher une équivalence entre deux termes.
				\medbreak
				Cependant on ajoute une $\delta$-réduction en plus qui va s'occuper de gérer les réductions avec opérations.
				\bigbreak
				\newpage
			\subsection{Les différentes machines traitées}\label{Machine}
				\subsubsection{CC Machine}
					
					CC vient des termes \textbf{Control string} et \textbf{Context} qui représente respectivement:
					\begin{itemize}
						\item la partie du $\lambda$-calcul que l'on traite
						\item la partie du $\lambda$-calcul que l'on met en attente 
					\end{itemize}
					Elle utilise le language ISWIM.
					\bigbreak
				
					Les règles définit pour cette machine sont les suivantes :
					\begin{enumerate}
						\item $\langle(M$ $N),E\rangle \longmapsto_{cc} \langle M,E[([]$ $N)]\rangle$ $si M \notin V$
						\item $\langle(V_{1}$ $N),E\rangle \longmapsto_{cc}  \langle M,E[(V_{1}$ $[])]\rangle$ $ si M \notin V$
						\item $\langle(o^{n}$ $V_{1}...V_{i}$ $M$ $N ...),E\rangle \longmapsto_{cc}  
						\langle M,E[(o^{n}$ $V_{1}...V_{i}$ $[]$ $N ...)]\rangle$ $ si M \notin V$
						\item $\langle((\lambda X.M) V),E\rangle \longmapsto_{cc} \langle M[X\longleftarrow V],E\rangle$
						\item $\langle(o^{n}$ $b_{1}...b_{n}),E\rangle \longmapsto_{cc}  \langle V,E\rangle$ avec $V = \delta(o^{n},b_{1}...b_{n})$ 
						\item $\langle V,E[(U$ $[])]\rangle \longmapsto_{cc} \langle(U$ $V),E\rangle$ 
						\item $\langle V,E[[]$ $N)]\rangle \longmapsto_{cc} \langle(V$ $N),E\rangle$ 
						\item $\langle V,E[(o^{n}$ $V_{1}...V_{i}$ $[]$ $N ...)]\rangle \longmapsto_{cc} \langle(o^{n}$ $V_{1}...V_{i}$ $V$ $N ...),E\rangle$ 
					\end{enumerate}
					\bigbreak
					
					La machine peut s'arrêter dans 3 états différents:
					\begin{itemize}
						\item[]$\longrightarrow$ on a une \textbf{constante b} tels que $\langle M,[]\rangle \twoheadrightarrow_{cc} \langle b,[]\rangle$;
						\item[]$\longrightarrow$ on a une \textbf{abstraction function} tels que $\langle M,[]\rangle \twoheadrightarrow_{cc} \langle\lambda X.N,[]\rangle$;
						\item[]$\longrightarrow$ on a un \textbf{état inconnu} soit une \textbf{erreur}.
					\end{itemize}
					\hyperref[CC]{Un exemple de la machine CC est fait dans les Annexes.}
					\bigbreak
					
					
				\subsubsection{SCC Machine}
					Le SCC est une simplification de règle du CC. En effet, le CC exploite uniquement les informations de la chaîne de contrôle (Control string). Du coup on combine certaines règles pour en faire qu'une.
					\bigbreak
					
					Les règles qui définisse la machine SCC sont les suivantes :
					
					\begin{enumerate}
							\item $\langle(M$ $N),E\rangle \longmapsto_{scc} \langle M,E[([]$ $N)]\rangle$
							\item $\langle(o^{n}$ $M$ $N...),E\rangle \longmapsto_{scc} \langle M,E[(o^{n}$ $[]$ $N...)]\rangle$
							\item $\langle V,E[((\lambda X.M)$ $[])]\rangle \longmapsto_{scc} \langle M[X\leftarrow V],E\rangle$
							\item $\langle V,E[[]$ $N)]\rangle \longmapsto_{scc} \langle N,E[(V$ $[])]\rangle$
							\item $\langle b,E[(o^{n},b_{1},...b_{i},[])]\rangle \longmapsto_{scc} \langle V,E\rangle$ avec $\delta (o^{n},b_{1},...b_{i},b)=V$ 
							\item $\langle V,E[(o^{n},V_{1},...V_{i},[],N$ $L)]\rangle \longmapsto_{scc} \langle N,E[(o^{n},V_{1},...V_{i},V,[],L)]\rangle$
					\end{enumerate}
					\bigbreak
					
					De même que pour la machine CC,la machine SCC peut s'arrêter dans 3 états différents:
					\begin{itemize}
						\item[]$\longrightarrow$ on a une \textbf{constante b} tels que $\langle M,[]\rangle \twoheadrightarrow_{scc} \langle b,[]\rangle$;
						\item[]$\longrightarrow$ on a une \textbf{abstraction function} tels que $\langle M,[]\rangle \twoheadrightarrow_{scc} \langle\lambda X.N,[]\rangle$;
						\item[]$\longrightarrow$ on a un \textbf{état inconnu} soit une \textbf{erreur}.
					\end{itemize}
					\bigbreak
					\hyperref[SCC]{Un exemple de la machine SCC est fait dans les Annexes.}
					\newpage
				\subsubsection{CK Machine}
					
					Les machines CC et SCC fonctionnent en allant chercher le plus à l'intérieur, c'est-à-dire que si l'on a une application on va en créer une intermédiaire dans le context avec un trou et traité la partie gauche de cette application etc jusqu'à arriver à un état traitable pour pouvoir "reconstruire", en reprenant l'application intermédiaire. C'est le style \textbf{ LIFO (Last In, First Out)}. Ce qui fait que les étapes de transition dépendent directement de la forme du 1ère élément et non de la structure générale.
					\smallbreak
					Pour palier à ce problème, la machine CK ajoute un nouvelle élément le \textbf{registre de contexte d'évaluation}, nommé $\kappa$, qui garde la partie "le plus à l'intérieur" accessible facilement.
					\smallbreak
					
					$\kappa$ = mt
					\begin{itemize}
						\item[|] $\langle fun,V,\kappa \rangle$
						\item[|] $\langle arg,N,\kappa \rangle$
						\item[|] $\langle opd,\langle V,...,V,o^{n}\rangle,\langle N,... \rangle,\kappa \rangle$
					\end{itemize}
					
					Cette structure est nommé \textbf{la continuation}.
					\medbreak
					
					Les règles qui définisse la machine CK sont les suivantes :
					\begin{enumerate}
						\item $\langle(M$ $N),\kappa\rangle \longmapsto_{ck} \langle M,\langle arg,N,\kappa\rangle\rangle$
						\item $\langle(o^{n}$ $M$ $N...),\kappa\rangle \longmapsto_{ck} \langle M,\langle opd,\langle o^{n}\rangle,\langle N,...\rangle,\kappa\rangle\rangle$
						\item $\langle V,\langle fun,(\lambda X.M),\kappa \rangle \rangle \longmapsto_{ck} \langle M[X \leftarrow V],\kappa\rangle$
						\item $\langle V,\langle arg,N,\kappa \rangle \rangle \longmapsto_{ck} \langle N,\langle fun,V,\kappa \rangle \rangle$
						\item $\langle b,\langle opd,\langle b_{i},...b_{1},o^{n}\rangle,\langle\rangle,\kappa\rangle\rangle \longmapsto_{ck} \langle V,\kappa\rangle$ avec $\delta(o^{n},b_{1},...b_{i},b) = V$
						\item $\langle V,\langle opd,\langle V',...o^{n}\rangle,\langle N,L,...\rangle,\kappa\rangle\rangle \longmapsto_{ck} \langle N,\langle opd,\langle V,V',...o^{n}\rangle,\langle L,...\rangle,\kappa\rangle\rangle$
					\end{enumerate}
					\bigbreak
					
					la machine CK peut s'arrêter dans 3 états différents:
					\begin{itemize}
						\item[]$\longrightarrow$ on a une \textbf{constante b} tels que $\langle M,mt\rangle \twoheadrightarrow_{ck} \langle b,mt\rangle$;
						\item[]$\longrightarrow$ on a une \textbf{abstraction function} tels que $\langle M,mt\rangle \twoheadrightarrow_{ck} \langle\lambda X.N,mt\rangle$;
						\item[]$\longrightarrow$ on a un \textbf{état inconnu} soit une \textbf{erreur}.
					\end{itemize}
				
					\hyperref[CK]{Un exemple de la machine CK est fait dans les Annexes.}
					\bigbreak
				
				\subsubsection{CEK Machine}
					Pour toutes les machines vues pour l'instant la $\beta$-réduction était appliquée immédiatement. Cela coûte cher surtout quand l'expression devient grande. De plus, si notre substitution n'est pas une variable elle est traité avant d'être appliqué.
					\smallbreak
					Il est plus intéressant d'appliquer les substitutions quand on en a vraiment la necessitée. Pour cela, la machine CEK ajoute les clauses et un environnement $\varepsilon$ qui va stocker les substitutions à faire.
					\medbreak
					
					On a alors:
					\begin{itemize}
						\item[] $\varepsilon$ = une fonction $\{\langle X,c\rangle,...\}$    c = $\{\langle M,\varepsilon\rangle$ $|$ $FV(M)\subset dom(\varepsilon)\}$    v = $\{\langle V,\varepsilon\rangle$ $|$ $\langle V,\varepsilon\rangle \in c\}$
						\item[] $\varepsilon[X \leftarrow c]$ = $\{\langle X,c\rangle\}$ $\cup$ $\{\langle Y,c'\rangle$ $|$ $\langle Y,c'\rangle \in \varepsilon$ et $ Y \neq X\}$
					\end{itemize}
					\medbreak
					
					$\kappa$ est renommé $\overline{\kappa}$ et est définit par :
					\smallbreak
					$\overline{\kappa}$ = mt
					\begin{itemize}
						\item[|] $\langle fun,v,\overline{\kappa} \rangle$
						\item[|] $\langle arg,c,\overline{\kappa} \rangle$
						\item[|] $\langle opd,\langle v,...,v,o^{n}\rangle,\langle c,... \rangle,\overline{\kappa} \rangle$
					\end{itemize}
					\bigbreak
					
					
					Les règles qui définisse la machine CEK sont les suivantes :
					\begin{enumerate}
						\item $\langle\langle(M$ $N),\varepsilon\rangle,\overline{\kappa}\rangle \longmapsto_{cek} \langle \langle M,\varepsilon\rangle,\langle arg,\langle N,\varepsilon\rangle,\overline{\kappa}\rangle\rangle$
						\item $\langle\langle(o^{n}$ $M$ $N...),\varepsilon\rangle,\overline{\kappa}\rangle \longmapsto_{cek} \langle \langle M,\varepsilon\rangle,\langle opd,\langle o^{n}\rangle,\langle \langle N,\varepsilon\rangle,...\rangle,\overline{\kappa}\rangle\rangle$
						\item $\langle\langle V,\varepsilon\rangle,\langle fun,\langle (\lambda X1.M),\varepsilon'\rangle,\overline{\kappa} \rangle \rangle \longmapsto_{cek} \langle \langle M,\varepsilon'[X1 \leftarrow \langle V,\varepsilon\rangle]\rangle,\overline{\kappa}\rangle$ si $V \notin X$
						\item  $\langle \langle V,\varepsilon\rangle,\langle arg,\langle N,\varepsilon'\rangle,\kappa\rangle\rangle \longmapsto_{cek} \langle \langle N,\varepsilon'\rangle,\langle fun,\langle V,\varepsilon\rangle,\overline{\kappa}\rangle\rangle$ si $V \notin X$
						\item $\langle  \langle b,\varepsilon\rangle,\langle opd,\langle \langle b_{i},\varepsilon_{i}\rangle,...\langle b_{1},\varepsilon_{1}\rangle ,o^{n}\rangle,\langle\rangle,\overline{\kappa}\rangle\rangle \longmapsto_{cek} \langle \langle V,\emptyset\rangle,\overline{\kappa}\rangle$ avec $\delta(o^{n},b_{1},...b_{i},b) = V$
						\item $\langle \langle V,\varepsilon\rangle,\langle opd,\langle v',...o^{n}\rangle,\langle  \langle N,\varepsilon'\rangle,c,...\rangle,\overline{\kappa}\rangle\rangle \longmapsto_{cek} \langle \langle N,\varepsilon'\rangle,\langle opd,\langle  \langle V,\varepsilon\rangle,v',...o^{n}\rangle,\langle c,...\rangle,\overline{\kappa}\rangle\rangle$ si $V \notin X$	
						\item $\langle\langle X,\varepsilon\rangle,\overline{\kappa}\rangle \longmapsto_{cek} \langle c,\overline{\kappa}\rangle$ avec $\varepsilon(X) = c$
						
					\end{enumerate}
					\bigbreak
					
					la machine CEK peut s'arrêter dans 3 états différents:
					\begin{itemize}
						\item[]$\longrightarrow$ on a une \textbf{constante b} tels que $\langle \langle M,\emptyset\rangle,mt\rangle \twoheadrightarrow_{cek} \langle\langle b,\varepsilon\rangle,mt\rangle$;
						\item[]$\longrightarrow$ on a une \textbf{abstraction function} tels que $\langle \langle M,\emptyset\rangle,mt\rangle \twoheadrightarrow_{cek} \langle\langle \lambda X.N,\varepsilon\rangle,mt\rangle$;
						\item[]$\longrightarrow$ on a un \textbf{état inconnu} soit une \textbf{erreur}.
					\end{itemize}
					\bigbreak
				
					
					\hyperref[CEK]{Un exemple de la machine CEK est fait dans les Annexes.}
					
					\newpage
					
				\subsubsection{SECD Machine}
					La différence entre la machine CEK et SECD est la façon dont le contexte est sauvegarder pendant que les sous-expressions sont évaluées.\\
					En effet, dans la machine SECD le contexte est créer par un appel de fonction, quand toute est stocké dans $\widehat{D}$ pour laisser un espace de travail. Par contre pour la machine CEK, le contexte est créé quant on évalue une application ou un argument indépendamment de la complexité de celui-ci.
					\smallbreak 
					Dans les langages tels que Java, Pascal ou encore C la façon de faire de la machine SECD est plus naturel. Par contre  dans les langages $\lambda$-calculs, Scheme ou encore ML c'est la façon de faire de la machine CEK qui est la plus naturel.
					\medbreak
					
					La machine SECD est composé d'une pile ($\widehat{S}$), d'un environnement ($\widehat{\varepsilon}$), d'une chaîne de contrôle ($\widehat{C}$) et d'une sauvegarde ($\widehat{D}$). Les différentes définitions de ces élément sont les suivantes :
					\smallbreak
					\begin{itemize}
						\item[] $\widehat{S}$ = $\epsilon$ | $\widehat{V}$ $\widehat{S}$ 
						\item[] $\widehat{\varepsilon}$ = une fonction $\{\langle X,\widehat{V}\rangle,...\}$
						\item[] $\widehat{C}$ = $\epsilon$ | b $\widehat{C}$ | X $\widehat{C}$ | ap $\widehat{C}$ | $prim_{o^{n}}$ $\widehat{C}$ | $\langle X,\widehat{C}\rangle$ $\widehat{C}$
						\item[] $\widehat{D}$ = $\epsilon$ | $\langle\widehat{S},\widehat{\varepsilon},\widehat{C},\widehat{D}\rangle$
						\item[] $\widehat{V}$ = b | $\langle\langle X,\widehat{C}\rangle,\widehat{\varepsilon}\rangle$
						\item[] $\textlbrackdbl b\textrbrackdbl _{secd}$ = b
						\item[] $\textlbrackdbl X\textrbrackdbl _{secd}$ = X
						\item[] $\textlbrackdbl (M_{1}$ $M_{2})\textrbrackdbl _{secd}$ = $\textlbrackdbl M_{1}\textrbrackdbl _{secd}$ $\textlbrackdbl M_{2}\textrbrackdbl _{secd}$ ap
						\item[] $\textlbrackdbl (o^{n}$ $M_{1}...M_{n})\textrbrackdbl _{secd}$ = $\textlbrackdbl M_{1}\textrbrackdbl _{secd}$ $...$ $\textlbrackdbl M_{n}\textrbrackdbl _{secd}$ $prim_{o^{n}}$
						\item[]  $\textlbrackdbl(\lambda X.M)\textrbrackdbl _{secd}$ =  $\langle X,\textlbrackdbl M\textrbrackdbl _{secd}\rangle$
					\end{itemize}
					\bigbreak
					
					Les règles qui définisse la machine SECD sont les suivantes :
					
					\begin{enumerate}
						\item $\langle\widehat{S},\widehat{\varepsilon},b$ $\widehat{C},\widehat{D}\rangle \longmapsto_{secd} \langle b$ $\widehat{S},\widehat{\varepsilon},\widehat{C},\widehat{D}\rangle$
						\item $\langle\widehat{S},\widehat{\varepsilon},X$ $\widehat{C},\widehat{D}\rangle \longmapsto_{secd} \langle \widehat{V}$ $\widehat{S},\widehat{\varepsilon},\widehat{C},\widehat{D}\rangle$ où $\widehat{V} = \varepsilon(X)$
						\item $\langle b_{1}$ $...$ $b_{n}\widehat{S},\widehat{\varepsilon},prim_{o^{n}}$ $\widehat{C},\widehat{D}\rangle \longmapsto_{secd} \langle \widehat{V}$ $\widehat{S},\widehat{\varepsilon},\widehat{C},\widehat{D}\rangle$ où $\widehat{V} = \delta(o^{n},b_1,...b_{n})$
						\item $\langle\widehat{S},\widehat{\varepsilon},\langle X,C'\rangle$ $\widehat{C},\widehat{D}\rangle \longmapsto_{secd} \langle\langle\langle X,C'\rangle,\varepsilon\rangle$ $\widehat{S},\widehat{\varepsilon},\widehat{C},\widehat{D}\rangle$
						\item $\langle\widehat{V}$ $\langle\langle X,C'\rangle,\varepsilon'\rangle$ $\widehat{S},\widehat{\varepsilon},ap$ $\widehat{C},\widehat{D}\rangle \longmapsto_{secd} \langle\epsilon,\varepsilon'[X \leftarrow \widehat{V}],C',\langle\widehat{S},\widehat{\varepsilon},\widehat{C},\widehat{D}\rangle\rangle$
						\item $\langle\widehat{V}$ $\widehat{S},\widehat{\varepsilon},\emptyset,\langle\widehat{S'},\widehat{\varepsilon'},\widehat{C'},\widehat{D}\rangle\rangle \longmapsto_{secd} \langle \widehat{V}$ $\widehat{S'},\widehat{\varepsilon'},\widehat{C'},\widehat{D}\rangle$
					\end{enumerate}
					\bigbreak
					
					la machine SECD peut s'arrêter dans 3 états différents:
					\begin{itemize}
						\item[]$\longrightarrow$ on a une \textbf{constante b} tels que $\langle \epsilon,\emptyset,\textlbrackdbl M\textrbrackdbl_{secd},\epsilon\rangle \twoheadrightarrow_{secd} \langle b,\widehat{\varepsilon},\epsilon,\epsilon\rangle$;
						\item[]$\longrightarrow$ on a une \textbf{abstraction function} tels que $\langle \epsilon,\emptyset,\textlbrackdbl M\textrbrackdbl_{secd},\epsilon\rangle \twoheadrightarrow_{secd} \langle\langle\langle X,\widehat{C}\rangle,\widehat{\varepsilon'}\rangle,\widehat{\varepsilon},\epsilon,\epsilon\rangle$;
						\item[]$\longrightarrow$ on a un \textbf{état inconnu} soit une \textbf{erreur}.
					\end{itemize}
					\bigbreak
					
					
					\hyperref[SECD]{Un exemple de la machine SECD est fait dans les Annexes.}
					
					\newpage
		\section{Dernière version de la machine SECD concurrente}\label{SECDConc4}
			La machine ayant beaucoup évoluée depuis le début du travail, je vais redéfinir chaque élément de la machine pour pouvoir mieux comprendre les nouvelles règles.
			\bigbreak
			
			Une suite de fonctions ont été écrite pour simplifier la lecture des règles. Les voici :
			\begin{itemize}
				\item[] $\rho(l,v,s,i) =$ la fonction qui pour une liste des signaux l, une valeur v, un signal s et un identifiant du thread courant donnés, renvoie la liste l' avec v ajouté à la liste des valeurs du signal s pour le thread i.
				\\\textbf{Exemple :} $\rho(\{...,\langle s,\{...,\langle id,valeur\rangle,...\},emit\rangle,...\},v,s,id) = \{...,\langle s,\{...,\langle id,valeur$ $v\rangle,...\},emit\rangle,...\}$
				\item[] 
				\item[] $\gamma(l,s,i,i') =$ la fonction qui pour une liste de valeurs partagées classés par signals et par thread l, un signal s, l'identifiant du thread courant et l'identifiant du thread auquels on veut accédé, renvoie soit un couple la liste avec l'itérateur déplacé et la valeur ou une exception si on ne peut plus donner de nouvelles valeurs 
				\\\textbf{Exemple :}
				\\$\gamma(\{...,\langle s,\{...,\langle id,\{...,\langle b,\{...,id',...\}\rangle,\langle n,\{...\}\rangle,...\},\{...\}\rangle,...\}\rangle,...\},s,id,id') = 
				\\\langle b , \{...,\langle s,\{...,\langle id,\{...,\langle b,\{...\}\rangle,\langle n,\{...,id'\}\rangle,...\},\{...\}\rangle,...\}\rangle,...\} \rangle$
				\medbreak
				$\gamma(\{...,\langle s,\{...,\langle id,\{...,\langle b,\{...,id',...\}\rangle\},\{...\}\rangle,...\}\rangle,...\},s,id,id') = 
				\\\langle b , \{...,\langle s,\{...,\langle id,\{...,\langle b,\{...\}\rangle\},\{...,id'\}\rangle,...\}\rangle,...\} \rangle$
				\medbreak
				$\gamma(\{...,\langle s,\{...,\langle id,valeurs,\{...,id',...\}\rangle,...\}\rangle,...\},s,id,id') = throw$ $erreur_{e}$
				\item[]
				\item[] $\iota(l,s,i) =$ la fonction qui pour une liste de signaux courant l, un signal s, renvoie une liste des signaux courant avec le signal s initialisé.
				\\\textbf{Exemple :} $\iota(\{...\},s) = \{...,\langle s,\{\},false\rangle\}$
				\item[]
				\item[] $\beta(l,s) =$ la fonction qui pour une liste de signal courant l et un signal s donnés, renvoie le booléen émit.
				\\\textbf{Exemple :}
				\\ $\beta(\{...,\langle s,\{...\},vraie\rangle,...\},s) = vraie$\\
				$\beta(\{...,\langle s,\{...\},faux\rangle,...\},s) = faux$
				\item[] 
				\item[] $\varepsilon(l,s) =$ la fonction qui pour une liste de signaux courant l et un signal s donnés, renvoie la liste avec le booléen représentant l'émission du signal à vraie.
				\\\textbf{Exemple :}
				\\ $\varepsilon(\{...,\langle s,\{...\},faux\rangle,...\},s) = \{...,\langle s,\{...\},vraie\rangle,...\}$
				\item[] 
			\end{itemize}
			\newpage
			Soit $\langle I,S,E,C,TL,SI,D,H,IP\rangle$ avec :
			
			\begin{itemize}
				\item[] $V = b$	
				\begin{itemize}
					\item[|] $\langle\langle X,C' \rangle E\rangle$
					\item[|] $erreur_{e}$
				\end{itemize}
				\item[] $I =$ un entier représentant l'identifiant du thread
				\item[] $S =  \emptyset$ 
				\begin{itemize}
					\item[|] $V S$
					\item[|] $signal$ $S$
					\item[|] $throw$ $S$
				\end{itemize}
				\item[] $E = \{...,\langle X,V\rangle,...\}$
				\item[] $C = \epsilon$
				\begin{itemize}
					\item[|] $b$ $C$~~~~~~~~~~~~~~~~~~~~~~~~~~    (une constante)
					\item[|] $X$ $C$~~~~~~~~~~~~~~~~~~~~~~~~~~(une variable)
					\item[|] $signal$ $C$~~~~~~~~~~~~~~~~~~~~(un signal)
					\item[|] $\langle X,C'\rangle$ $C$~~~~~~~~~~~~~~~~~~~(une abstraction)
					\item[|] $ap$ $C$~~~~~~~~~~~~~~~~~~~~~~~~~(une application)
					\item[|] $prim_{o^{n}}$ $C$~~~~~~~~~~~~~~~~~~~(un opérateur)
					\item[|] $bspawn$ $C$~~~~~~~~~~~~~~~~~~~(début d'un nouveau thread)
					\item[|] $espawn$ $C$~~~~~~~~~~~~~~~~~~(fin d'un nouveau thread)
					\item[|] $\langle C',C''\rangle$ $C$~~~~~~~~~~~~~~~~~~(le test de présence d'un signal)
					\item[|] $emit$ $C$~~~~~~~~~~~~~~~~~~~~~~(émet un signal)
					\item[|] $init$ $C$~~~~~~~~~~~~~~~~~~~~~~~(initialise un signal pour une chaîne de contrôle donné)
					\item[|] $put$ $C$~~~~~~~~~~~~~~~~~~~~~~~~(insère une valeur dans la liste de valeurs d'un signal)
					\item[|] $get$ $C$~~~~~~~~~~~~~~~~~~~~~~~~(prends une valeurs dans la liste de valeurs d'un signal)
					\item[|] $erreur_{e}$ $C$~~~~~~~~~~~~~~~~~~(une erreur)
					\item[|] $throw$ $C$~~~~~~~~~~~~~~~~~~~~(lève une erreur)
					\item[|] $\langle e,\langle C'\langle X,C''\rangle\rangle\rangle$ $C$~~~~~~~(un gestionnaire d'erreur)
				\end{itemize}
				\item[] TL = $\langle W,ST\rangle$
				\begin{itemize}
					\item[] W = $\{...,\langle I,S,E,C,D\rangle,...\}$~~~~~~~~(liste des threads en attente)
					\item[] ST = $\{...,\langle s,\langle I,S,E,C,D\rangle\rangle,...\}$~~(liste des threads en attente d'un signal)
				\end{itemize}
				\item[] SI = $\langle CS,SSI\rangle$
				\begin{itemize}
					\item[] $CS = \{...,\langle s,\{...,\langle id,\{...,b,...\}\rangle,...\},emit\rangle,...\}$~~~~~~~~~~~~~~~~~~~~~~~~~~(liste des signaux courants)
					\begin{itemize}
						\item [] on va découper cette élément pour mieux en comprendre le sens :
						\item[] - $\{...,*,...\}$ Une liste. 
						\item[] - $\langle s,\{...,**,...\},emit\rangle$ \\
						Une liste composée de trinôme comportant le identifiant du signal, une sous-liste et un booléen exprimant l'émission de ce signal.
						\item[] - $\langle id,\{...,b,...\}\rangle$ \\
						Une sous-liste composée d'un trinôme comportant l'identifiant du thread et une liste de valeur.
					\end{itemize}
					\item[]
					\item[] $SSI =  \{...,\langle s,\{...,\langle id,\{...,\langle b,\{...,id',...\}\rangle,...\},\{...,id'',...\}\rangle,...\}\rangle,...\}$~~~~~(liste des signaux partagés)
					\begin{itemize}
						\item[] comme pour CS on va découper cette élément pour pouvoir le comprendre :
						\item[] - $\{...,*,...\}$ Une liste. 
						\item[] - $\langle s,\{...,**,...\}\rangle$\\
						Une liste composée d'un couple comportant un identifiant de signal et d'une sous-liste
						\item[] - $\langle id,\{...,***,...\},\{...,id'',...\}\rangle$\\
						Une sous-liste composée d'un trinôme comportant un identifiant d'un thread, d'un liste et d'une sous-sous-liste d'identifiant de thread représentant la liste des threads ayant fini leurs parcours de la sous-sous-liste.
						\item[] - $\langle b,\{...,id',...\}\rangle$\\
						Une sous-sous-liste composée d'un couple comportant une valeur et une liste d'identifiant de threads qui représente un pointeur
					\end{itemize}
				\end{itemize}
				\item[] D = $\emptyset$
				\begin{itemize}
					\item[|] $\langle S,E,C,D\rangle$~~~~~~~~~(une sauvegarde liée à une abstraction)
				\end{itemize}
				\item[] H = $\emptyset$ 
				\begin{itemize}
					\item[|] $\langle e\langle I,S,E,\langle X,C' \rangle C,TL,SI,D,H,IP\rangle\rangle$
				\end{itemize} 
				\item[] IP = un entier servant à attribuer l'identifiant à un nouveau thread 
			\end{itemize}
		
			\newpage
			
			Les éléments étant expliqués, voici les nouvelles règles de la machine :
			\medbreak
			\begin{enumerate}
				\item[] \textbf{Partie de base de la machine SECD} 
				\item[]
				\begin{itemize}
					\item[] \textbf{Constante :} On a une constante, on la déplace dans la pile.
					\\$\langle I,S,E,b$ $C,TL,SI,D,H,IP\rangle 
					\longrightarrow_{secdv4} \langle I,b$ $S,E,C,TL,SI,D,H,IP\rangle$
					\item[]
					
					\item[] \textbf{Substitution :} On a une abstraction, on créer une fermeture avec celle-ci et l'environnement courant et on la place dans la pile.
					\\$\langle I,S,E,X$ $C,TL,SI,D,H,IP\rangle
					\longrightarrow_{secdv4} \langle I,V$ $S,E,C,TL,SI,D,H,IP\rangle$\\ avec E(X) = V
					\item[]
					
					\item[] \textbf{Opération :} On a un opérateur et le nombre de constante nécessaire dans la pile, \\via la fonction $\delta$ on retourne le résultat dans la pile.
					\\ $\langle I,b_{n},...,b_{1}$ $S,E,prim_{o^{n}}$ $C,TL,SI,D,H,IP\rangle
					\longrightarrow_{secdv4} \langle I,V$ $S,E,C,TL,SI,D,H,IP\rangle$
					\\avec $\delta(o^{n}$ $b_{1}...b_{n}) = V$
					\item[]
					
					\item[]  \textbf{Abstraction :} On a une abstraction, on créer une fermeture comportant l'abstraction et l'environnement courant et on mets la fermeture dans la pile.
					\\$\langle I,S,E,\langle X,C'\rangle$ $C,TL,SI,D,H,IP\rangle
					\longrightarrow_{secdv4} \langle I,\langle\langle X,C'\rangle,E\rangle$ $S,E,C,TL,SI,D,H,IP\rangle$
					\item[]
					
					\item[]  \textbf{Application :} On a une application, donc on sauvegarde dans le dépôt, on ajoute une substitution et on remplace la chaîne de contrôle et l'environnement par ceux présent dans la fermeture.
					\\$\langle I,V$ $\langle\langle X,C'\rangle,E'\rangle$ $S,E,ap$ $C,TL,SI,D,H,IP\rangle
					\longrightarrow_{secdv4} \langle I,\emptyset,E'[X \leftarrow V],C',TL,SI,\langle S,E,C,D\rangle,H,IP\rangle$
					\item[] 
					
					\item[] \textbf{Récupération de sauvegarde :}  On a rien mais le dépôt comporte une sauvegarde donc on prends celle-ci.
					\\  $\langle I,V$ $S,E,\epsilon,TL,SI,\langle S',E',C,D\rangle,H,IP\rangle
					\longrightarrow_{secdv4} \langle I,V$ $S',E',C,TL,SI,D,H,IP\rangle$
				\end{itemize}
				\item[]
			
			    
			    
			    \item[] \textbf{Partie pour les erreurs} 
			    \item[]
			    \begin{itemize}
			    	\item[] \textbf{Erreur :} On a une erreur, on la déplace en tête de la pile.
			    	\\$\langle I,S,E,erreur_{e}$ $C,TL,SI,D,H,IP\rangle
			    	\longrightarrow_{secdv4} \langle I,erreur_{e}$ $S,E,C,TL,SI,D,H,IP\rangle$
			    	\item[]
			    	
			    	\item[] \textbf{Opération sur erreur :} On a l'opérateur qui traite cette erreur donc on mets le résultat de 
			    	\\la fonction $\delta$ dans la pile.
			    	\\$\langle I,throw$ $erreur_{e}$ $S,E,prim_{o^{1_{e}}}$ $C,TL,SI,D,H,IP\rangle
			    	\longrightarrow_{secdv4} \langle I,V$ $S,E,C,TL,SI,D,H,IP\rangle$
			    	\\ avec $\delta(o^{1_{e}}$ $erreur_{e}) = V$
			    	\item[]
			    	
			    	\item[] \textbf{Propagation :} On a un un élément excepté l'opérateur qui traite cette erreur donc on propage l'erreur.
			    	\\ $\langle I,throw$ $erreur_{e}$ $S,E,M$ $C,TL,SI,D,H,IP\rangle
			    	\longrightarrow_{secdv4} \langle I,throw$ $erreur_{e}$ $S,E,C,TL,SI,D,H,IP\rangle$ 
			    	\\avec $M = $ un élément de $C$ $\backslash$ $prim_{o^{1_{e}}}$
			    	\item[]
			    	
			    	\item[] \textbf{Traitée erreur via gestionnaire d'erreur :} On a plus rien mais on a une erreur levé dans la pile du coup 
			    	\\ on regarde si le gestionnaire d'erreur gère celle-ci ; oui du coup prend la sauvegarde.
			    	\\ $\langle I,throw$ $erreur_{e}$ $S,E,\epsilon,TL,SI,D,\langle e,\langle I',S',E',\langle X,C'' \rangle C',TL',SI',D',H,IP'\rangle\rangle,IP\rangle \\\longrightarrow_{secdv4} \langle I',\emptyset,E'[X \leftarrow erreur_{e}],C'',TL',SI',\langle S',E',C',D'\rangle,H,IP'\rangle$
			    	\item[]
			    	
			    	\item[]  \textbf{Traitement erreur récursif :} On a plus rien mais on a une erreur levé dans la pile du coup on regarde 
			    	\\si le gestionnaire d'erreur gère celle-ci mais non du coup on regarde pour le gestionnaire sauvegardé.
			    	\\ $\langle I,throw$ $erreur_{e}$ $S,E,\epsilon,TL,SI,D,\langle e',\langle I',S',E',\langle X,C'' \rangle C',TL',SI',D',H,IP'\rangle\rangle,IP\rangle \\\longrightarrow_{secdv4}\langle I,throw$ $erreur_{e}$ $S,E,\epsilon,TL,SI,D,H,IP\rangle$
			    	\item[]
			    	\newpage
			    	
			    	\item[]  \textbf{Création d'un gestionnaire d'erreur :} On a un try...catch donc on test avec la chaîne de contrôle du try et on sauvegarde catch dans le gestionnaire d'erreur.
			    	\\ $\langle I,erreur_{e}$ $S,E,\langle C',\langle X,C''\rangle\rangle$ $C,TL,SI,D,H,IP\rangle 
			    	\\\longrightarrow_{secdv4} \langle I,S,E,C'$ $C,TL,SI,D,\langle e,\langle I,erreur_{e}$ $S,E,\langle X,C'' \rangle$ $C,TL,SI,D,H,IP\rangle\rangle,IP\rangle$
			    \end{itemize}
			    
			    \item[]
			
			    
			 
			    
			    \item[] \textbf{Partie pour la concurrence} 
			    \item[]
			    \begin{itemize}
			    	\item[]  \textbf{Création thread :} On veut créer un nouveau thread.
			    	\\ $\langle I,S,E,bspawn$ $C'$ $espawn$ $C,\langle W,ST\rangle,SI,D,H,IP\rangle 
			    	\\\longrightarrow_{secdv4} \langle I,S,E,C,\langle W$ $\langle IP,S,E,C',D\rangle,ST\rangle,SI,D,H,IP+1\rangle$
			    	\item[]
			    	
			    	\item[]  \textbf{Signal :} On a un signal, on le déplace dans la pile.
			    	\\ $\langle I,S,E,signal$ $C,TL,SI,D,H,IP\rangle
			    	\longrightarrow_{secdv4} \langle I,signal$ $S,E,C,TL,SI,D,H,IP\rangle$
			    	\item[]
			    	
			    	\item[] \textbf{Ajouter dans un signal :} On ajoute une constante dans une liste de valeurs d'un signal via la fonction $\rho$
			    	\\$\langle I,s$ $b$ $S,E,put$ $C,TL,\langle CS,SSI\rangle,D,H,IP\rangle \longrightarrow_{secdv4} \langle I,S,E,C,TL,\langle CS',SSI\rangle,D,H,IP\rangle$ \\
			    	avec $CS' = \rho(CS,b,s,I)$
			    	\item[]
			    	
			    	\item[] \textbf{Prendre une valeur partagée (possible) :} On prends dans la liste de valeurs d'un signal partagé lié \\
			    	à un identifant une constante via la fonction $\gamma$ .
			    	\\$\langle I,s$ $b$ $\langle\langle X,C'\rangle,E'\rangle$ $S,E,get$ $C,TL,\langle CS,SSI\rangle,D,H,IP\rangle 
			    	\\\longrightarrow_{secdv4} \langle I,\emptyset,E'[X \leftarrow V],C',TL,\langle CS,SSI'\rangle,\langle S,E,C,D\rangle,H,IP\rangle$\\
			    	avec $ \gamma(SSI,s,I,b) = \langle V,SSI'\rangle$ si il reste une valeur à prendre
			    	\item[]
			    	
			    	\item[] \textbf{Prendre une valeur partagée (impossible) :} On prends dans la liste de valeurs d'un signal partagé lié à un identifant une constante via la fonction $\gamma$ mais on a déjà tout pris donc on lève une erreur.
			    	\\ $\langle I,s$ $b$ $\langle\langle X,C'\rangle,E'\rangle$ $S,E,get$ $C,TL,\langle CS,SSI\rangle,D,H,IP\rangle 
			    	\\\longrightarrow_{secdv4} \langle I,throw$ $erreur_{e}$ $S,E,C,TL,\langle CS,SSI'\rangle,D,H,IP\rangle$\\
			    	avec $ \gamma(SSI,s,I,b) = throw$ $erreur_{e}$ si il reste aucune valeur à prendre
			    	\item[]
			    	
			    	\item[] \textbf{Initialisation signal :} On initialise le signal via la fonction $\iota$.
			    	\\ $\langle I,s$ $S,E,init$ $C,TL,\langle CS,SSI\rangle,D,H,IP\rangle 
			    	\longrightarrow_{secdv4} \langle I,S,E,C,TL,\langle CS',SSI\rangle,D,H,IP\rangle$\\
			    	avec $\iota(CS,s) = CS'$
			    	\item[]	
			    	
			    	\item[] \textbf{Conditionnelle signal :} On teste la présence d'un signal, via la fonction $\beta$ on sait qu'il est émit donc on prends le 1er choix.
			    	\\ $\langle I,s$ $S,E,\langle C',C''\rangle$ $C,TL,\langle CS,SSI\rangle,D,H,IP\rangle 
			    	\longrightarrow_{secdv4} \langle I,S,E,C'$ $C,TL,\langle CS,SSI\rangle,D,H,IP\rangle$ \\
			    	avec $\beta(CS,s) = vraie$
			    	\item[]
			    	
			    	
			    	\item[] \textbf{ Thread bloqué remplacé :} On teste la présence d'un signal, via la fonction $\beta$ on sait qu'il n'est pas émit et il y a un thread dans la file d'attente
			    	donc on mets ce thread dans la liste de threads bloqués et on prends le thread en tête de la file.
			    	\\$\langle I,s$ $S,E,\langle C',C''\rangle$ $C,\langle\langle I',S',E',C''',D'\rangle W,ST\rangle,\langle CS,SSI\rangle,D,H,IP\rangle 
			    	\\\longrightarrow_{secdv4} \langle I',S',E',C''',\langle W,ST\langle s,\langle I,s$ $S,E,\langle C',C''\rangle$ $C,D\rangle\rangle\rangle,\langle CS,SSI\rangle,D',H,IP\rangle$ \\
			    	avec $\beta(CS,s) = faux$
			    	\item[]	
			    	
			    	\item[] \textbf{Thread bloqué non remplacé :} On teste la présence d'un signal, via la fonction $\beta$ on sait qu'il n'est pas émit donc on mets ce thread dans la liste de threads bloqués.
			    	\\ $\langle I,s$ $S,E,\langle C',C''\rangle$ $C,\langle \emptyset,ST\rangle,\langle CS,SSI\rangle,D,H,IP\rangle 
			    	\\\longrightarrow_{secdv4} \langle IP,\emptyset,\epsilon,\emptyset,\langle W,ST\langle s,\langle I,s$ $S,E,\langle C',C''\rangle$ $C,D\rangle\rangle\rangle,\langle CS,SSI\rangle,\emptyset,H,IP+1\rangle$ \\
			    	avec $\beta(CS,s) = faux$
			    	\item[]	
			    	
			    	\item[] \textbf{Émission :} On émet un signal donc on mets dans la file d'attente tous les threads attendant le signal.
			    	\\ $\langle I,s$ $S,E,emit$ $C,TL,\langle CS,SSI\rangle,D,H,IP\rangle 
			    	\longrightarrow_{secdv4} \langle I,Unit$ $S,E,C,TL',\langle CS',SSI\rangle,D,H,IP\rangle$\\
			    	avec $\varepsilon(CS,s) = CS'$ et TL' = $\langle W',ST'\rangle$ et TL = $\langle W,ST\rangle$ :
			    	\begin{itemize}
			    		\item[] W' = W $\cup$ les éléments de ST qui attendent le signal s 
			    		\item[] ST' = ST $\setminus$ les éléments de ST qui attendent le signal s 
			    	\end{itemize}
			    	\item[]
			    	
			    	\item[] \textbf{Récupération dans la file d'attente :} On a plus rien à traité et on a aucune sauvegarde, du coup 
			    	\\on change le thread courant par le thread en tête de la file d'attente.
			    	\\$\langle I,V$ $S,E,\epsilon,\langle\langle I',S',E',C,D\rangle W,ST\rangle,SI,\emptyset,H,IP\rangle 
			    	\longrightarrow_{secdv4} \langle I',V$ $S',E',C,\langle W,ST\rangle,SI,D,H,IP\rangle$
			    	\item[]
			    	
			    	\item[] \textbf{Fin d'instant logique :} On a plus rien à traiter, on a aucune sauvegarde et on a plus rien dans la file d'attente, c'est la fin d'un instant logique.
			    	\\$\langle I,V$ $S,E,\epsilon,\langle\emptyset,ST\rangle,SI,\emptyset,H,IP\rangle 
			    	\longrightarrow_{secdv4} \langle I,V$ $S,E,\epsilon,\langle W,\emptyset\rangle,SI',\emptyset,H,IP\rangle$\\
			    	avec W = ST avec tous ces éléments qui prennent en compte l'absence de l'émission du signal attendu
			    	\\ SI' = $\langle CS',SSI'\rangle$ et SI = $\langle CS,SSI\rangle$
			    	\begin{itemize}
			    		\item[] CS' = CS avec les listes de valeurs remise à zéro ainsi que le booléen représentant l'émission est mis à faux
			    		\item[] SSI' = SSI avec tous les signaux émit à l'instant qui vient de finir et leurs listes de valeurs associé
			    	\end{itemize}
			    	\item[]
			    	
		    	\end{itemize}
			    \item[] \textbf{Partie commune} 
			   	\item[]
			   	\begin{itemize}
			   		\item[] \textbf{Application neutre :} On a une application sur rien, cela revient juste à rien faire.
			     	\\ $\langle I,S,E,ap$ $C,TL,SI,D,H,IP\rangle
			   		\longrightarrow_{secdv4} \langle I,S,E,C,TL,SI,D,H,IP\rangle$
			   	\end{itemize}
			    
				\item[]
			\end{enumerate}
		
			\bigbreak
			\bigbreak
			la machine SECD version 4 peut s'arrêter dans 3 états différents:
			\begin{itemize}
				\item[] on a une \textbf{constante b} tels que $\langle 0,\emptyset,\emptyset,\textlbrackdbl M\textrbrackdbl_{secdv4},\langle\emptyset,\emptyset\rangle,\langle\emptyset,\emptyset\rangle,\emptyset,\emptyset,\emptyset,1\rangle \\
				\twoheadrightarrow_{secdv4} \langle I,b$ $S,E,\epsilon,\langle\emptyset,\emptyset\rangle,SI,\emptyset,H,IP\rangle$;
				\item[] 
				\item[] on a une \textbf{abstraction function} tels que
				$\langle 0,\emptyset,\emptyset,\textlbrackdbl M\textrbrackdbl_{secdv4},\langle\emptyset,\emptyset\rangle,\langle\emptyset,\emptyset\rangle,\emptyset,\emptyset,\emptyset,1\rangle \\
				\twoheadrightarrow_{secdv4} \langle I,\langle\langle X,C\rangle,E'\rangle$ $S,E,\epsilon,\langle\emptyset,\emptyset\rangle,SI,\emptyset,H,IP\rangle$;
				\item[] 
				\item[] on a une \textbf{erreur e} tels que 
				$\langle 0,\emptyset,\emptyset,\textlbrackdbl M\textrbrackdbl_{secdv4},\langle\emptyset,\emptyset\rangle,\langle\emptyset,\emptyset\rangle,\emptyset,\emptyset,\emptyset,1\rangle \\
				\twoheadrightarrow_{secdv4} \langle I,erreur_{e}$ $S,E,\epsilon,\langle\emptyset,\emptyset\rangle,SI,\emptyset,H,IP\rangle$;
			\end{itemize}
		\newpage
		\section{Conclusion}
		
		\newpage
		
		\section{Annexes}
			\subsection{Les Exemples des machines étudiées}
				\subsubsection{Exemple de fonctionnement de la machine CC}\label{CC}
			
					Voici un exemple de fonctionnement de la machine CC:
					
					\begin{itemize}
						\item[] CC machine : $\langle(((\lambda f.\lambda x.f$ $x)$ $\lambda y.(+$ $y$ $y))$ $\ulcorner 1\urcorner),[]\rangle$
						\item[] > $\langle(M$ $N),E\rangle \longmapsto_{cc} \langle M,E[([]$ $N)]\rangle$ $si M \notin V$
						\item[] CC machine : $\langle((\lambda f.\lambda x.f$ $x)$ $\lambda y.(+$ $y$ $y)),[([]$ $\ulcorner 1\urcorner)]\rangle$
						\item[] > $\langle((\lambda X.M) V),E\rangle \longmapsto_{cc} \langle M[X\longleftarrow V],E\rangle$
						\item[] CC machine : $\langle(\lambda x.f$ $x)[f \leftarrow \lambda y.(+$ $y$ $y)],[([]$ $\ulcorner 1\urcorner)]\rangle$
						\item[] CC machine : $\langle(\lambda x.(\lambda y.(+$ $y$ $y))$ $x),[([]$ $\ulcorner 1\urcorner)]\rangle$
						\item[] > $\langle V,E[[]$ $N)]\rangle \longmapsto_{cc} \langle(V$ $N),E\rangle$ 
						\item[] CC machine : $\langle((\lambda x.(\lambda y.(+$ $y$ $y))$ $x)$ $\ulcorner 1\urcorner),[]\rangle$
						\item[] > $\langle((\lambda X.M) V),E\rangle \longmapsto_{cc} \langle M[X\longleftarrow V],E\rangle$
						\item[] CC machine : $\langle((\lambda y.(+$ $y$ $y))$ $x)[x \leftarrow \ulcorner 1\urcorner],[]\rangle$
						\item[] CC machine : $\langle((\lambda y.(+$ $y$ $y))$ $\ulcorner 1\urcorner),[]\rangle$
						\item[] > $\langle((\lambda X.M) V),E\rangle \longmapsto_{cc} \langle M[X\longleftarrow V],E\rangle$
						\item[] CC machine : $\langle(+$ $y$ $y)[y \leftarrow \ulcorner 1\urcorner],[]\rangle$
						\item[] CC machine : $\langle(+$ $\ulcorner 1\urcorner$ $\ulcorner 1\urcorner),[]\rangle$
						\item[] > $\langle(o^{n}$ $b_{1}...b_{n}),E\rangle \longmapsto_{cc}  \langle V,E\rangle$ avec $V = \delta(o^{n},b_{1}...b_{n})$
						\item[] CC machine : $\langle\ulcorner 2\urcorner,[]\rangle$
					\end{itemize}
				
				\subsubsection{Exemple de fonctionnement de la machine SCC}\label{SCC}
				
					Voici un exemple de fonctionnement de la machine SCC :
					
					\begin{itemize}
						\item[] SCC machine : $\langle(((\lambda f.\lambda x.f$ $x)$ $\lambda y.(+$ $y$ $y))$ $\ulcorner 1\urcorner),[]\rangle$
						\item[] > $\langle(M$ $N),E\rangle \longmapsto_{scc} \langle M,E[([]$ $N)]\rangle$
						\item[] SCC machine : $\langle((\lambda f.\lambda x.f$ $x)$ $\lambda y.(+$ $y$ $y)),[([]$ $\ulcorner 1\urcorner)]\rangle$
						\item[] > $\langle(M$ $N),E\rangle \longmapsto_{scc} \langle M,E[([]$ $N)]\rangle$
						\item[] SCC machine : $\langle(\lambda f.\lambda x.f$ $x),[([]$ $\ulcorner 1\urcorner),([]$ $(\lambda y.(+$ $y$ $y)))]\rangle$
						\item[] >  $\langle V,E[[]$ $N)]\rangle \longmapsto_{scc} \langle N,E[(V$ $[])]\rangle$
						\item[] SCC machine : $\langle(\lambda y.(+$ $y$ $y)),[([]$ $\ulcorner 1\urcorner),((\lambda f.\lambda x.f$ $x)$ $[])]\rangle$
						\item[] > $\langle V,E[((\lambda X.M)$ $[])]\rangle \longmapsto_{scc} \langle M[X\leftarrow V],E\rangle$
						\item[] SCC machine : $\langle(\lambda x.f$ $x)[f\leftarrow(\lambda y.(+$ $y$ $y))],[([]$ $\ulcorner 1\urcorner)]\rangle$
						\item[] SCC machine : $\langle(\lambda x.(\lambda y.(+$ $y$ $y))$ $x),[([]$ $\ulcorner 1\urcorner)]\rangle$
						\item[] >  $\langle V,E[[]$ $N)]\rangle \longmapsto_{scc} \langle N,E[(V$ $[])]\rangle$
						\item[] SCC machine : $\langle\ulcorner 1\urcorner,[((\lambda x.(\lambda y.(+$ $y$ $y))$ $x)$ $[])]\rangle$
						\item[] > $\langle V,E[((\lambda X.M)$ $[])]\rangle \longmapsto_{scc} \langle M[X\leftarrow V],E\rangle$
						\item[] SCC machine : $\langle((\lambda y.(+$ $y$ $y))$ $x)[x \leftarrow\ulcorner 1\urcorner],[]\rangle$
						\item[] SCC machine : $\langle((\lambda y.(+$ $y$ $y))$ $\ulcorner 1\urcorner,[]\rangle$
						\item[] > $\langle(M$ $N),E\rangle \longmapsto_{scc} \langle M,E[([]$ $N)]\rangle$
						\item[] SCC machine : $\langle(\lambda y.(+$ $y$ $y)),[([]$ $\ulcorner 1\urcorner)]\rangle$
						\item[] >  $\langle V,E[[]$ $N)]\rangle \longmapsto_{scc} \langle N,E[(V$ $[])]\rangle$
						\item[] SCC machine : $\langle\ulcorner 1\urcorner,[(\lambda y.(+$ $y$ $y))$ $[])]\rangle$
						\item[] > $\langle V,E[((\lambda X.M)$ $[])]\rangle \longmapsto_{scc} \langle M[X\leftarrow V],E\rangle$
						\item[] SCC machine : $\langle(+$ $y$ $y)[y \leftarrow \ulcorner 1\urcorner],[]\rangle$	
						\item[] SCC machine : $\langle(+$ $\ulcorner 1\urcorner$ $\ulcorner 1\urcorner),[]\rangle$	
						\item[] > $\langle(o^{n}$ $M$ $N...),E\rangle \longmapsto_{scc} \langle M,E[(o^{n}$ $[]$ $N...)]\rangle$
						\item[] SCC machine : $\langle\ulcorner 1\urcorner,(+$ $[]$ $\ulcorner 1\urcorner)\rangle$	
						\item[] > $\langle V,E[(o^{n},V_{1},...V_{i},[],N$ $L)]\rangle \longmapsto_{scc} \langle N,E[(o^{n},V_{1},...V_{i},V,[],L)]\rangle$
						\item[] SCC machine : $\langle\ulcorner 1\urcorner,(+$ $\ulcorner 1\urcorner$ $[])\rangle$	
						\item[] $\langle b,E[(o^{n},b_{1},...b_{i},[])]\rangle \longmapsto_{scc} \langle V,E\rangle$ avec $\delta (o^{n},b_{1},...b_{i},b)=V$ 
						\item[] SCC machine : $\langle\ulcorner 2\urcorner,[]\rangle$	
					\end{itemize}
				\newpage
				\subsubsection{Exemple de fonctionnement de la machine CK}\label{CK}
				
					Voici un exemple de fonctionnement de la machine CK :
					
					\begin{itemize}
						\item[] CK machine : $\langle(((\lambda f.\lambda x.f$ $x)$ $\lambda y.(+$ $y$ $y))$ $\ulcorner 1\urcorner),mt\rangle$
						\item[] > $\langle(M$ $N),\kappa\rangle \longmapsto_{ck} \langle M,\langle arg,N,\kappa\rangle\rangle$
						\item[] CK machine : $\langle((\lambda f.\lambda x.f$ $x)$ $\lambda y.(+$ $y$ $y)),\langle arg,\ulcorner 1\urcorner,mt\rangle\rangle$
						\item[] > $\langle(M$ $N),\kappa\rangle \longmapsto_{ck} \langle M,\langle arg,N,\kappa\rangle\rangle$
						\item[] CK machine : $\langle(\lambda f.\lambda x.f$ $x),\langle arg,(\lambda y.(+$ $y$ $y)),\langle arg,\ulcorner 1\urcorner,mt\rangle\rangle\rangle$	
						\item[] > $\langle V,\langle arg,N,\kappa \rangle \rangle \longmapsto_{ck} \langle N,\langle fun,V,\kappa \rangle \rangle$
						\item[] CK machine : $\langle(\lambda y.(+$ $y$ $y)),\langle fun,(\lambda f.\lambda x.f$ $x),\langle arg,\ulcorner 1\urcorner,mt\rangle\rangle\rangle$	
						\item[] > $\langle V,\langle fun,(\lambda X.M),\kappa \rangle \rangle \longmapsto_{ck} \langle M[X \leftarrow V],\kappa\rangle$
						\item[] CK machine : $\langle(\lambda x.f$ $x)[f\leftarrow(\lambda y.(+$ $y$ $y))],\langle arg,\ulcorner 1\urcorner,mt\rangle\rangle$	
						\item[] CK machine : $\langle(\lambda x.(\lambda y.(+$ $y$ $y))$ $x),\langle arg,\ulcorner 1\urcorner,mt\rangle\rangle$	
						\item[] > $\langle V,\langle arg,N,\kappa \rangle \rangle \longmapsto_{ck} \langle N,\langle fun,V,\kappa \rangle \rangle$
						\item[] CK machine : $\langle\ulcorner 1\urcorner,\langle fun,(\lambda x.(\lambda y.(+$ $y$ $y))$ $x),mt\rangle\rangle$
						\item[] > $\langle V,\langle fun,(\lambda X.M),\kappa \rangle \rangle \longmapsto_{ck} \langle M[X \leftarrow V],\kappa\rangle$		
						\item[] CK machine : $\langle((\lambda y.(+$ $y$ $y))$ $x)[x \leftarrow \ulcorner 1\urcorner ],mt\rangle$
						\item[] CK machine : $\langle((\lambda y.(+$ $y$ $y))$ $\ulcorner 1\urcorner),mt\rangle$
						\item[] > $\langle(M$ $N),\kappa\rangle \longmapsto_{ck} \langle M,\langle arg,N,\kappa\rangle\rangle$
						\item[] CK machine : $\langle(\lambda y.(+$ $y$ $y)),\langle arg,\ulcorner 1\urcorner,mt\rangle\rangle$	
						\item[] > $\langle V,\langle arg,N,\kappa \rangle \rangle \longmapsto_{ck} \langle N,\langle fun,V,\kappa \rangle \rangle$
						\item[] CK machine : $\langle\ulcorner 1\urcorner,\langle fun,(\lambda y.(+$ $y$ $y)),mt\rangle\rangle$	
						\item[] > $\langle V,\langle fun,(\lambda X.M),\kappa \rangle \rangle \longmapsto_{ck} \langle M[X \leftarrow V],\kappa\rangle$	
						\item[] CK machine : $\langle(+$ $y$ $y)[y \leftarrow\ulcorner 1\urcorner],mt\rangle$
						\item[] CK machine : $\langle(+$ $\ulcorner 1\urcorner$ $\ulcorner 1\urcorner),mt\rangle$
						\item[] > $\langle(o^{n}$ $M$ $N...),\kappa\rangle \longmapsto_{ck} \langle M,\langle opd,\langle o^{n}\rangle,\langle N,...\rangle,\kappa\rangle\rangle$
						\item[] CK machine : $\langle\ulcorner 1\urcorner,\langle opd,\langle + \rangle,\langle\ulcorner 1\urcorner\rangle,mt\rangle\rangle$
						\item[] > $\langle V,\langle opd,\langle V',...o^{n}\rangle,\langle N,L,...\rangle,\kappa\rangle\rangle \longmapsto_{ck} \langle N,\langle opd,\langle V,V',...o^{n}\rangle,\langle L,...\rangle,\kappa\rangle\rangle$
						\item[] CK machine : $\langle\ulcorner 1\urcorner,\langle opd,\langle\ulcorner 1\urcorner,+ \rangle,\langle\rangle,mt\rangle\rangle$
						\item[] > $\langle b,\langle opd,\langle b_{i},...b_{1},o^{n}\rangle,\langle\rangle,\kappa\rangle\rangle \longmapsto_{ck} \langle V,\kappa\rangle$ avec $\delta(o^{n},b_{1},...b_{i},b) = V$
						\item[] CK machine : $\langle\ulcorner 2\urcorner,mt\rangle$
					\end{itemize}
				\newpage
				
				\subsubsection{Exemple de fonctionnement de la machine CEK}\label{CEK}
					
					Voici un exemple de fonctionnement de la machine CEK :
					
					\begin{itemize}
						\item[] CEK machine : $\langle\langle(((\lambda f.\lambda x.f$ $x)$ $\lambda y.(+$ $y$ $y))$ $\ulcorner 1\urcorner)\emptyset\rangle,mt\rangle$
						\item[] > $\langle\langle(M$ $N),\varepsilon\rangle,\overline{\kappa}\rangle \longmapsto_{cek} \langle \langle M,\varepsilon\rangle,\langle arg,\langle N,\varepsilon\rangle,\overline{\kappa}\rangle\rangle$
						\item[] CEK machine : $\langle\langle((\lambda f.\lambda x.f$ $x)$ $\lambda y.(+$ $y$ $y))\emptyset\rangle,\langle arg,\langle\ulcorner 1\urcorner,\emptyset\rangle,mt\rangle\rangle$
						\item[] > $\langle\langle(M$ $N),\varepsilon\rangle,\overline{\kappa}\rangle \longmapsto_{cek} \langle \langle M,\varepsilon\rangle,\langle arg,\langle N,\varepsilon\rangle,\overline{\kappa}\rangle\rangle$
						\item[] CEK machine : $\langle\langle(\lambda f.\lambda x.f$ $x),\emptyset\rangle,\langle arg,\langle(\lambda y.(+$ $y$ $y)),\emptyset\rangle,\langle arg,\langle\ulcorner 1\urcorner,\emptyset\rangle,mt\rangle\rangle\rangle$
						\item[] >  $\langle \langle V,\varepsilon\rangle,\langle arg,\langle N,\varepsilon'\rangle,\kappa\rangle\rangle \longmapsto_{cek} \langle \langle N,\varepsilon'\rangle,\langle fun,\langle V,\varepsilon\rangle,\overline{\kappa}\rangle\rangle$ si $V \notin X$
						\item[] CEK machine : $\langle\langle(\lambda y.(+$ $y$ $y)),\emptyset\rangle,\langle fun,\langle(\lambda f.\lambda x.f$ $x),\emptyset\rangle,\langle arg,\langle\ulcorner 1\urcorner,\emptyset\rangle,mt\rangle\rangle\rangle$
						\item[] > $\langle\langle V,\varepsilon\rangle,\langle fun,\langle (\lambda X1.M),\varepsilon'\rangle,\overline{\kappa} \rangle \rangle \longmapsto_{cek} \langle \langle M,\varepsilon'[X1 \leftarrow \langle V,\varepsilon\rangle]\rangle,\overline{\kappa}\rangle$ si $V \notin X$
						\item[] CEK machine : $\langle\langle(\lambda x.f$ $x),\emptyset[f \leftarrow \langle(\lambda y.(+$ $y$ $y)),\emptyset\rangle]\rangle,\langle arg,\langle\ulcorner 1\urcorner,\emptyset\rangle,mt\rangle\rangle$
						\item[] CEK machine : $\langle\langle(\lambda x.f$ $x),\{\langle f,\langle(\lambda y.(+$ $y$ $y)),\emptyset\rangle\rangle\}\rangle,\langle arg,\langle\ulcorner 1\urcorner,\emptyset\rangle,mt\rangle\rangle$
						\item[] >  $\langle \langle V,\varepsilon\rangle,\langle arg,\langle N,\varepsilon'\rangle,\kappa\rangle\rangle \longmapsto_{cek} \langle \langle N,\varepsilon'\rangle,\langle fun,\langle V,\varepsilon\rangle,\overline{\kappa}\rangle\rangle$ si $V \notin X$
						\item[] CEK machine : $\langle\langle\ulcorner 1\urcorner,\emptyset\rangle,\langle fun,\langle(\lambda x.f$ $x),\{\langle f,\langle(\lambda y.(+$ $y$ $y)),\emptyset\rangle\rangle\}\rangle,mt\rangle\rangle$
						\item[] > $\langle\langle V,\varepsilon\rangle,\langle fun,\langle (\lambda X1.M),\varepsilon'\rangle,\overline{\kappa} \rangle \rangle \longmapsto_{cek} \langle \langle M,\varepsilon'[X1 \leftarrow \langle V,\varepsilon\rangle]\rangle,\overline{\kappa}\rangle$ si $V \notin X$
						\item[] CEK machine : $\langle\langle(f$ $x),\{\langle f,\langle(\lambda y.(+$ $y$ $y)),\emptyset\rangle\rangle\}[x \leftarrow \langle\ulcorner 1\urcorner,\emptyset\rangle]\rangle,mt\rangle$
						\item[] CEK machine : $\langle\langle(f$ $x),\{\langle f,\langle(\lambda y.(+$ $y$ $y)),\emptyset\rangle\rangle,\langle x,\langle\ulcorner 1\urcorner,\emptyset\rangle\rangle\}\rangle,mt\rangle$
						\item[] > $\langle\langle(M$ $N),\varepsilon\rangle,\overline{\kappa}\rangle \longmapsto_{cek} \langle \langle M,\varepsilon\rangle,\langle arg,\langle N,\varepsilon\rangle,\overline{\kappa}\rangle\rangle$
						\item[] CEK machine : $\langle\langle f,\{\langle f,\langle(\lambda y.(+$ $y$ $y)),\emptyset\rangle\rangle,\langle x,\langle\ulcorner 1\urcorner,\emptyset\rangle\rangle\}\rangle,\langle arg,\langle x,\{\langle f,\langle(\lambda y.(+$ $y$ $y)),\emptyset\rangle\rangle,\langle x,\langle\ulcorner 1\urcorner,\emptyset\rangle\rangle\}\rangle,mt\rangle\rangle$
						\item[] > $\langle\langle X,\varepsilon\rangle,\overline{\kappa}\rangle \longmapsto_{cek} \langle c,\overline{\kappa}\rangle$ avec $\varepsilon(X) = c$
						\item[] CEK machine : $\langle\langle(\lambda y.(+$ $y$ $y)),\emptyset\rangle,\langle arg,\langle x,\{\langle f,\langle(\lambda y.(+$ $y$ $y)),\emptyset\rangle\rangle,\langle x,\langle\ulcorner 1\urcorner,\emptyset\rangle\rangle\}\rangle,mt\rangle\rangle$
						\item[] > $\langle \langle V,\varepsilon\rangle,\langle arg,\langle N,\varepsilon'\rangle,\kappa\rangle\rangle \longmapsto_{cek} \langle \langle N,\varepsilon'\rangle,\langle fun,\langle V,\varepsilon\rangle,\overline{\kappa}\rangle\rangle$ si $V \notin X$
						\item[] CEK machine : $\langle\langle x,\{\langle f,\langle(\lambda y.(+$ $y$ $y)),\emptyset\rangle\rangle,\langle x,\langle\ulcorner 1\urcorner,\emptyset\rangle\rangle\}\rangle,\langle fun,\langle(\lambda y.(+$ $y$ $y)),\emptyset\rangle,mt\rangle\rangle$	
						\item[] > $\langle\langle X,\varepsilon\rangle,\overline{\kappa}\rangle \longmapsto_{cek} \langle c,\overline{\kappa}\rangle$ avec $\varepsilon(X) = c$
						\item[] CEK machine : $\langle \langle\ulcorner 1\urcorner,\emptyset\rangle,\langle fun,\langle(\lambda y.(+$ $y$ $y)),\emptyset\rangle,mt\rangle\rangle$	
						\item[] > $\langle\langle V,\varepsilon\rangle,\langle fun,\langle (\lambda X1.M),\varepsilon'\rangle,\overline{\kappa} \rangle \rangle \longmapsto_{cek} \langle \langle M,\varepsilon'[X1 \leftarrow \langle V,\varepsilon\rangle]\rangle,\overline{\kappa}\rangle$ si $V \notin X$
						\item[] CEK machine : $\langle\langle(+$ $y$ $y),\emptyset[y \leftarrow\langle\ulcorner 1\urcorner,\emptyset\rangle]\rangle ,mt\rangle$
						\item[] CEK machine : $\langle\langle(+$ $y$ $y),\{\langle y,\langle\ulcorner 1\urcorner,\emptyset\rangle\rangle\}\rangle ,mt\rangle$
						\item[] > $\langle\langle(o^{n}$ $M$ $N...),\varepsilon\rangle,\overline{\kappa}\rangle \longmapsto_{cek} \langle \langle M,\varepsilon\rangle,\langle opd,\langle o^{n}\rangle,\langle \langle N,\varepsilon\rangle,...\rangle,\overline{\kappa}\rangle\rangle$
						\item[] CEK machine : $\langle\langle y,\{\langle y,\langle\ulcorner 1\urcorner,\emptyset\rangle\rangle\}\rangle ,\langle opd,\langle+\rangle,\langle\langle y,\{\langle y,\langle\ulcorner 1\urcorner,\emptyset\rangle\rangle\}\rangle\rangle,mt\rangle\rangle$
						\item[] > $\langle\langle X,\varepsilon\rangle,\overline{\kappa}\rangle \longmapsto_{cek} \langle c,\overline{\kappa}\rangle$ avec $\varepsilon(X) = c$
						\item[] CEK machine : $\langle\langle\ulcorner 1\urcorner,\emptyset\rangle,\langle opd,\langle+\rangle,\langle\langle y,\{\langle y,\langle\ulcorner 1\urcorner,\emptyset\rangle\rangle\}\rangle\rangle,mt\rangle\rangle$
						\item[] > $\langle \langle V,\varepsilon\rangle,\langle opd,\langle v',...o^{n}\rangle,\langle  \langle N,\varepsilon'\rangle,c,...\rangle,\overline{\kappa}\rangle\rangle \longmapsto_{cek} \langle \langle N,\varepsilon'\rangle,\langle opd,\langle  \langle V,\varepsilon\rangle,v',...o^{n}\rangle,\langle c,...\rangle,\overline{\kappa}\rangle\rangle$ si $V \notin X$
						\item[] CEK machine : $\langle\langle y,\{\langle y,\langle\ulcorner 1\urcorner,\emptyset\rangle\rangle\}\rangle,\langle opd,\langle\ulcorner 1\urcorner$ $+\rangle,\langle\rangle,mt\rangle\rangle$
						\item[] > $\langle\langle X,\varepsilon\rangle,\overline{\kappa}\rangle \longmapsto_{cek} \langle c,\overline{\kappa}\rangle$ avec $\varepsilon(X) = c$
						\item[] CEK machine : $\langle\langle\ulcorner 1\urcorner,\emptyset\rangle,\langle opd,\langle\ulcorner 1\urcorner$ $+\rangle,\langle\rangle,mt\rangle\rangle$
						\item[] > $\langle  \langle b,\varepsilon\rangle,\langle opd,\langle \langle b_{i},\varepsilon_{i}\rangle,...\langle b_{1},\varepsilon_{1}\rangle ,o^{n}\rangle,\langle\rangle,\overline{\kappa}\rangle\rangle \longmapsto_{cek} \langle \langle V,\emptyset\rangle,\overline{\kappa}\rangle$ avec $\delta(o^{n},b_{1},...b_{i},b) = V$
						\item[] CEK machine : $\langle\langle\ulcorner 2\urcorner,\emptyset\rangle,mt\rangle$
					\end{itemize}
				\newpage
				
				\subsubsection{Exemple de fonctionnement de la machine SECD}\label{SECD}
				
					Voici un exemple de fonctionnement de la machine SECD :	
					
					\begin{itemize}
						\item[] Conversion : $\textlbrackdbl(((\lambda f.\lambda x.f$ $x)$ $\lambda y.(+$ $y$ $y))$ $\ulcorner 1\urcorner)\textrbrackdbl_{secd}$
						\item[] Conversion : $\textlbrackdbl((\lambda f.\lambda x.f$ $x)$ $\lambda y.(+$ $y$ $y))\textrbrackdbl_{secd}$ $\textlbrackdbl\ulcorner 1\urcorner\textrbrackdbl_{secd}$ $ap$
						\item[] Conversion : $\textlbrackdbl(\lambda f.\lambda x.f$ $x)\textrbrackdbl_{secd}$ $\textlbrackdbl\lambda y.(+$ $y$ $y)\textrbrackdbl_{secd}$ $ap$ $\ulcorner 1\urcorner$ $ap$
						\item[] Conversion : $\langle f,\textlbrackdbl\lambda x.(f$ $x)\textrbrackdbl_{secd}\rangle$ $\langle y,y$ $y$ $prim_{+}\rangle$ $ap$ $\ulcorner 1\urcorner$ $ap$
						\item[] Conversion : $\langle f,\langle x,\textlbrackdbl f$ $x\textrbrackdbl_{secd}\rangle\rangle$ $\langle y,y$ $y$ $prim_{+}\rangle$ $ap$ $\ulcorner 1\urcorner$ $ap$
						\item[] Conversion : $\langle f,\langle x,f$ $x$ $ap\rangle\rangle$ $\langle y,y$ $y$ $prim_{+}\rangle$ $ap$ $\ulcorner 1\urcorner$ $ap$
						\item[] SECD Machine : $\langle\epsilon,\emptyset,\langle f,\langle x,f$ $x$ $ap\rangle\rangle$ $\langle y,y$ $y$ $prim_{+}\rangle$ $ap$ $\ulcorner 1\urcorner$ $ap,\epsilon\rangle$ 
						\item[] > $\langle\widehat{S},\widehat{\varepsilon},\langle X,C'\rangle$ $\widehat{C},\widehat{D}\rangle \longmapsto_{secd} \langle\langle\langle X,C'\rangle,\varepsilon\rangle$ $\widehat{S},\widehat{\varepsilon},\widehat{C},\widehat{D}\rangle$
						\item[] SECD Machine : $\langle\langle\langle f,\langle x,f$ $x$ $ap\rangle\rangle,\emptyset\rangle,\emptyset,\langle y,y$ $y$ $prim_{+}\rangle$ $ap$ $\ulcorner 1\urcorner$ $ap,\epsilon\rangle$ 
						\item[] > $\langle\widehat{S},\widehat{\varepsilon},\langle X,C'\rangle$ $\widehat{C},\widehat{D}\rangle \longmapsto_{secd} \langle\langle\langle X,C'\rangle,\varepsilon\rangle$ $\widehat{S},\widehat{\varepsilon},\widehat{C},\widehat{D}\rangle$
						\item[] SECD Machine : $\langle\langle\langle y,y$ $y$ $prim_{+}\rangle,\emptyset\rangle$ $\langle\langle f,\langle x,f$ $x$ $ap\rangle\rangle,\emptyset\rangle,\emptyset,ap$ $\ulcorner 1\urcorner$ $ap,\epsilon\rangle$
						\item[] > $\langle\widehat{V}$ $\langle\langle X,C'\rangle,\varepsilon'\rangle$ $\widehat{S},\widehat{\varepsilon},ap$ $\widehat{C},\widehat{D}\rangle \longmapsto_{secd} \langle\epsilon,\varepsilon'[X \leftarrow \widehat{V}],C',\langle\widehat{S},\widehat{\varepsilon},\widehat{C},\widehat{D}\rangle\rangle$
						\item[] SECD Machine : $\langle\epsilon,\emptyset[f \leftarrow \langle\langle y,y$ $y$ $prim_{+}\rangle,\emptyset\rangle],\langle x,f$ $x$ $ap\rangle,\langle\epsilon,\emptyset,\ulcorner 1\urcorner$ $ap,\epsilon\rangle\rangle$
						\item[] SECD Machine : $\langle\epsilon,\{\langle f,\langle\langle y,y$ $y$ $prim_{+}\rangle,\emptyset\rangle\rangle\},\langle x,f$ $x$ $ap\rangle,\langle\epsilon,\emptyset,\ulcorner 1\urcorner$ $ap,\epsilon\rangle\rangle$
						\item[] >  $\langle\widehat{S},\widehat{\varepsilon},\langle X,C'\rangle$ $\widehat{C},\widehat{D}\rangle \longmapsto_{secd} \langle\langle\langle X,C'\rangle,\varepsilon\rangle$ $\widehat{S},\widehat{\varepsilon},\widehat{C},\widehat{D}\rangle$
						\item[] SECD Machine : $\langle\langle\langle x,f$ $x$ $ap\rangle,\{\langle f,\langle\langle y,y$ $y$ $prim_{+}\rangle,\emptyset\rangle\rangle\}\rangle,\{f,\langle\langle y,y$ $y$ $prim_{+}\rangle,\emptyset\rangle\},\emptyset,\langle\epsilon,\emptyset,\ulcorner 1\urcorner$ $ap,\epsilon\rangle\rangle$
						\item[] > $\langle\widehat{V}$ $\widehat{S},\widehat{\varepsilon},\emptyset,\langle\widehat{S'},\widehat{\varepsilon'},\widehat{C'},\widehat{D}\rangle\rangle \longmapsto_{secd} \langle \widehat{V}$ $\widehat{S'},\widehat{\varepsilon'},\widehat{C'},\widehat{D}\rangle$
						\item[] SECD Machine : $\langle\langle\langle x,f$ $x$ $ap\rangle,\{\langle f,\langle\langle y,y$ $y$ $prim_{+}\rangle,\emptyset\rangle\rangle\}\rangle,\emptyset,\ulcorner 1\urcorner$ $ap,\epsilon\rangle$
						\item[] >  $\langle\widehat{S},\widehat{\varepsilon},b$ $\widehat{C},\widehat{D}\rangle \longmapsto_{secd} \langle b$ $\widehat{S},\widehat{\varepsilon},\widehat{C},\widehat{D}\rangle$
						\item[] SECD Machine :  $\langle\ulcorner 1\urcorner$ $\langle\langle x,f$ $x$ $ap\rangle,\{\langle f,\langle\langle y,y$ $y$ $prim_{+}\rangle,\emptyset\rangle\rangle\}\rangle,\emptyset,ap,\epsilon\rangle$
						\item[] >  $\langle\widehat{V}$ $\langle\langle X,C'\rangle,\varepsilon'\rangle$ $\widehat{S},\widehat{\varepsilon},ap$ $\widehat{C},\widehat{D}\rangle \longmapsto_{secd} \langle\epsilon,\varepsilon'[X \leftarrow \widehat{V}],C',\langle\widehat{S},\widehat{\varepsilon},\widehat{C},\widehat{D}\rangle\rangle$
						\item[] SECD Machine : $\langle\epsilon,\{\langle f,\langle\langle y,y$ $y$ $prim_{+}\rangle,\emptyset\rangle\rangle\}[x \leftarrow \ulcorner 1\urcorner],f$ $x$ $ap,\langle\epsilon,\emptyset,\epsilon,\epsilon\rangle\rangle$
						\item[] SECD Machine : $\langle\epsilon,\{\langle f,\langle\langle y,y$ $y$ $prim_{+}\rangle,\emptyset\rangle\rangle,\langle x,\ulcorner 1\urcorner\rangle\},f$ $x$ $ap,\langle\epsilon,\emptyset,\epsilon,\epsilon\rangle\rangle$
						\item[] > $\langle\widehat{S},\widehat{\varepsilon},X$ $\widehat{C},\widehat{D}\rangle \longmapsto_{secd} \langle \widehat{V}$ $\widehat{S},\widehat{\varepsilon},\widehat{C},\widehat{D}\rangle$ où $\widehat{V} = \varepsilon(X)$
						\item[] SECD Machine : $\langle\langle\langle y,y$ $y$ $prim_{+}\rangle,\emptyset\rangle,\{\langle f,\langle\langle y,y$ $y$ $prim_{+}\rangle,\emptyset\rangle\rangle,\langle x,\ulcorner 1\urcorner\rangle\},x$ $ap,\langle\epsilon,\emptyset,\epsilon,\epsilon\rangle\rangle$
						\item[] > $\langle\widehat{S},\widehat{\varepsilon},X$ $\widehat{C},\widehat{D}\rangle \longmapsto_{secd} \langle \widehat{V}$ $\widehat{S},\widehat{\varepsilon},\widehat{C},\widehat{D}\rangle$ où $\widehat{V} = \varepsilon(X)$
						\item[] SECD Machine : $\langle\ulcorner 1\urcorner$ $\langle\langle y,y$ $y$ $prim_{+}\rangle,\emptyset\rangle,\{\langle f,\langle\langle y,y$ $y$ $prim_{+}\rangle,\emptyset\rangle\rangle,\langle x,\ulcorner 1\urcorner\rangle\},ap,\langle\epsilon,\emptyset,\epsilon,\epsilon\rangle\rangle$
						\item[] >  $\langle\widehat{V}$ $\langle\langle X,C'\rangle,\varepsilon'\rangle$ $\widehat{S},\widehat{\varepsilon},ap$ $\widehat{C},\widehat{D}\rangle \longmapsto_{secd} \langle\epsilon,\varepsilon'[X \leftarrow \widehat{V}],C',\langle\widehat{S},\widehat{\varepsilon},\widehat{C},\widehat{D}\rangle\rangle$
						\item[] SECD Machine :	$\langle\epsilon,\emptyset[y \leftarrow \ulcorner 1\urcorner],y$ $y$ $prim_{+},\langle\epsilon,\{\langle f,\langle\langle y,y$ $y$ $prim_{+}\rangle,\emptyset\rangle\rangle,\langle x,\ulcorner 1\urcorner\rangle\} ,\emptyset,\langle\epsilon,\emptyset,\epsilon,\epsilon\rangle\rangle\rangle$
						\item[] SECD Machine : $\langle\epsilon,\{\langle y,\ulcorner 1\urcorner\rangle\},y$ $y$ $prim_{+},\langle\epsilon,\{\langle f,\langle\langle y,y$ $y$ $prim_{+}\rangle,\emptyset\rangle\rangle,\langle x,\ulcorner 1\urcorner\rangle\} ,\emptyset,\langle\epsilon,\emptyset,\epsilon,\epsilon\rangle\rangle\rangle$
						\item[] > $\langle\widehat{S},\widehat{\varepsilon},X$ $\widehat{C},\widehat{D}\rangle \longmapsto_{secd} \langle \widehat{V}$ $\widehat{S},\widehat{\varepsilon},\widehat{C},\widehat{D}\rangle$ où $\widehat{V} = \varepsilon(X)$
						\item[] SECD Machine : $\langle\ulcorner 1\urcorner,\{\langle y,\ulcorner 1\urcorner\rangle\},y$ $prim_{+},\langle\epsilon,\{\langle f,\langle\langle y,y$ $y$ $prim_{+}\rangle,\emptyset\rangle\rangle,\langle x,\ulcorner 1\urcorner\rangle\} ,\emptyset,\langle\epsilon,\emptyset,\epsilon,\epsilon\rangle\rangle\rangle$
						\item[] > $\langle\widehat{S},\widehat{\varepsilon},X$ $\widehat{C},\widehat{D}\rangle \longmapsto_{secd} \langle \widehat{V}$ $\widehat{S},\widehat{\varepsilon},\widehat{C},\widehat{D}\rangle$ où $\widehat{V} = \varepsilon(X)$
						\item[] SECD Machine : $\langle\ulcorner 1\urcorner$ $\ulcorner 1\urcorner,\{\langle y,\ulcorner 1\urcorner\rangle\},prim_{+},\langle\epsilon,\{\langle f,\langle\langle y,y$ $y$ $prim_{+}\rangle,\emptyset\rangle\rangle,\langle x,\ulcorner 1\urcorner\rangle\} ,\emptyset,\langle\epsilon,\emptyset,\epsilon,\epsilon\rangle\rangle\rangle$
						\item[] > $\langle b_{1}$ $...$ $b_{n}\widehat{S},\widehat{\varepsilon},prim_{o^{n}}$ $\widehat{C},\widehat{D}\rangle \longmapsto_{secd} \langle \widehat{V}$ $\widehat{S},\widehat{\varepsilon},\widehat{C},\widehat{D}\rangle$ où $\widehat{V} = \delta(o^{n},b_1,...b_{n})$
						\item[] SECD Machine : $\langle\ulcorner 2\urcorner,\{\langle y,\ulcorner 1\urcorner\rangle\},\emptyset,\langle\epsilon,\{\langle f,\langle\langle y,y$ $y$ $prim_{+}\rangle,\emptyset\rangle\rangle,\langle x,\ulcorner 1\urcorner\rangle\} ,\emptyset,\langle\epsilon,\emptyset,\epsilon,\epsilon\rangle\rangle\rangle$
						\item[] > $\langle\widehat{V}$ $\widehat{S},\widehat{\varepsilon},\emptyset,\langle\widehat{S'},\widehat{\varepsilon'},\widehat{C'},\widehat{D}\rangle\rangle \longmapsto_{secd} \langle \widehat{V}$ $\widehat{S'},\widehat{\varepsilon'},\widehat{C'},\widehat{D}\rangle$
						\item[] SECD Machine : $\langle\ulcorner 2\urcorner,\{\langle f,\langle\langle y,y$ $y$ $prim_{+}\rangle,\emptyset\rangle\rangle,\langle x,\ulcorner 1\urcorner\rangle\} ,\emptyset,\langle\epsilon,\emptyset,\epsilon,\epsilon\rangle\rangle$
						\item[] > $\langle\widehat{V}$ $\widehat{S},\widehat{\varepsilon},\emptyset,\langle\widehat{S'},\widehat{\varepsilon'},\widehat{C'},\widehat{D}\rangle\rangle \longmapsto_{secd} \langle \widehat{V}$ $\widehat{S'},\widehat{\varepsilon'},\widehat{C'},\widehat{D}\rangle$
						\item[] SECD Machine : $\langle\ulcorner 2\urcorner,\emptyset,\epsilon,\epsilon\rangle$
					\end{itemize}
					\newpage
			
			\subsection{Les différentes versions faite pour rendre la machine SECD concurrente}
				\subsubsection{1ère version des règles de la machine SECD Concurrente}\label{SECDConc1}
					Soit $\langle\widehat{S},\widehat{\varepsilon},\widehat{C},\widehat{W},\widehat{ST},\widehat{SI},\widehat{D}\rangle$ avec :
					
					\begin{itemize}
						\item[] $\widehat{S}$ = ... | $\langle C',C''\rangle$ $\widehat{S}$ | Remp $\widehat{S}$
						\item[] $\widehat{\varepsilon}$ = ... 
						\item[] $\widehat{C}$ = ... 
						\begin{itemize}
							\item[|] bspawn $\widehat{C}$ 
							\item[|] espawn $\widehat{C}$
							\item[|] $\langle C',C''\rangle$ $\widehat{C}$
							\item[|] $present_{s}$ $\widehat{C}$
							\item[|] $emit_{s}$ $\widehat{C}$ 
						\end{itemize}
						\item[] $\widehat{W}$ = $\{\widehat{D},...\}$
						\item[] $\widehat{ST}$ = $\{\langle s,\widehat{D}\rangle,...\}$
						\item[] $\widehat{SI}$ = $\{ s,...\}$
						\item[] $\widehat{D}$ = ...
					\end{itemize}
					\bigbreak
					
					
					Les nouvelles règles sont les suivantes :
					\smallbreak
					\begin{enumerate}
					\item $\langle\widehat{S},\widehat{\varepsilon},b$ $\widehat{C},\widehat{W},\widehat{ST},\widehat{SI},\widehat{D}\rangle \longmapsto_{secdv1} \langle b$ $\widehat{S},\widehat{\varepsilon},\widehat{C},\widehat{W},\widehat{ST},\widehat{SI},\widehat{D}\rangle$
					\item $\langle\widehat{S},\widehat{\varepsilon},X$ $\widehat{C},\widehat{W},\widehat{ST},\widehat{SI},\widehat{D}\rangle \longmapsto_{secdv1} \langle \widehat{V}$ $\widehat{S},\widehat{\varepsilon},\widehat{C},\widehat{W},\widehat{ST},\widehat{SI},\widehat{D}\rangle$ où $\widehat{V} = \varepsilon(X)$
					\item $\langle b_{1}$ $...$ $b_{n}\widehat{S},\widehat{\varepsilon},prim_{o^{n}}$ $\widehat{C},\widehat{W},\widehat{ST},\widehat{SI},\widehat{D}\rangle \longmapsto_{secdv1} \langle \widehat{V}$ $\widehat{S},\widehat{\varepsilon},\widehat{C},\widehat{W},\widehat{ST},\widehat{SI},\widehat{D}\rangle$ où $\widehat{V} = \delta(o^{n},b_1,...b_{n})$
					\item $\langle\widehat{S},\widehat{\varepsilon},\langle X,C'\rangle$ $\widehat{C},\widehat{W},\widehat{ST},\widehat{SI},\widehat{D}\rangle \longmapsto_{secdv1} \langle\langle\langle X,C'\rangle,\varepsilon\rangle$ $\widehat{S},\widehat{\varepsilon},\widehat{C},\widehat{W},\widehat{ST},\widehat{SI},\widehat{D}\rangle$
					\item $\langle\widehat{V}$ $\langle\langle X,C'\rangle,\varepsilon'\rangle$ $\widehat{S},\widehat{\varepsilon},ap$ $\widehat{C},\widehat{W},\widehat{ST},\widehat{SI},\widehat{D}\rangle \longmapsto_{secdv1} \langle\epsilon,\varepsilon'[X \leftarrow \widehat{V}],C',\widehat{W},\widehat{ST},\widehat{SI},\langle\widehat{S},\widehat{\varepsilon},\widehat{C},\widehat{D}\rangle\rangle$
					\item $\langle\widehat{V}$ $\widehat{S},\widehat{\varepsilon},\emptyset,\langle\widehat{S'},\widehat{\varepsilon'},\widehat{C'},\widehat{W},\widehat{ST},\widehat{SI},\widehat{D}\rangle\rangle \longmapsto_{secdv1} \langle \widehat{V}$ $\widehat{S'},\widehat{\varepsilon'},\widehat{C'},\widehat{W},\widehat{ST},\widehat{SI},\widehat{D}\rangle$
					\item $\langle\widehat{V}$ Remp $\widehat{S},\widehat{\varepsilon},ap$ $\widehat{C},\widehat{W},\widehat{ST},\widehat{SI},\widehat{D}\rangle \longmapsto_{secdv1} \langle \widehat{V}$ $\widehat{S},\widehat{\varepsilon},\widehat{C},\widehat{W},\widehat{ST},\widehat{SI},\widehat{D}\rangle$
					\item $\langle$ Remp $\widehat{V}\widehat{S},\widehat{\varepsilon},ap$ $\widehat{C},\widehat{W},\widehat{ST},\widehat{SI},\widehat{D}\rangle \longmapsto_{secdv1} \langle \widehat{V}$ $\widehat{S},\widehat{\varepsilon},\widehat{C},\widehat{W},\widehat{ST},\widehat{SI},\widehat{D}\rangle$
					\item $\langle$ Remp $\widehat{V}\widehat{S},\widehat{\varepsilon},ap$ $\widehat{C},\widehat{W},\widehat{ST},\widehat{SI},\widehat{D}\rangle \longmapsto_{secdv1} \langle \widehat{V}$ $\widehat{S},\widehat{\varepsilon},\widehat{C},\widehat{W},\widehat{ST},\widehat{SI},\widehat{D}\rangle$
					\item $\langle\widehat{S},\widehat{\varepsilon},bspawn$ $\widehat{C'}$ $espawn$ $\widehat{C},\widehat{W},\widehat{ST},\widehat{SI},\widehat{D}\rangle \longmapsto_{secdv1} \langle$ Remp $\widehat{S},\widehat{\varepsilon},\widehat{C},\widehat{W}\langle\widehat{S},\widehat{E},\widehat{C'},\widehat{D}\rangle,\widehat{ST},\widehat{SI},\widehat{D}\rangle$
					\item $\langle\widehat{S},\widehat{\varepsilon},\langle C',C''\rangle$ $\widehat{C},\widehat{W},\widehat{ST},\widehat{SI},\widehat{D}\rangle \longmapsto_{secdv1} \langle\langle C',C''\rangle$ $\widehat{S},\widehat{\varepsilon},\widehat{C},\widehat{W},\widehat{ST},\widehat{SI},\widehat{D}\rangle$
					\item $\langle\langle C',C''\rangle $ $\widehat{S},\widehat{\varepsilon},present_{s}$ $\widehat{C},\widehat{W},\widehat{ST},s$ $\widehat{SI},\widehat{D}\rangle \longmapsto_{secdv1} \langle\widehat{S},\widehat{\varepsilon},C'\widehat{C},\widehat{W},\widehat{ST},s$ $\widehat{SI},\widehat{D}\rangle$
					\item $\langle\langle C',C''\rangle $ $\widehat{S},\widehat{\varepsilon},present_{s}$ $\widehat{C},\langle\widehat{S'},\widehat{\varepsilon'},\widehat{C'''},\widehat{D'}\rangle$ $\widehat{W},\widehat{ST},\widehat{SI},\widehat{D}\rangle \longmapsto_{secdv1} \langle\widehat{S'},\widehat{\varepsilon'},\widehat{C'''},\widehat{W},\widehat{ST}\langle s,\langle\widehat{S},\widehat{\varepsilon},present_{s}$ $\widehat{C},\widehat{D}\rangle\rangle,\widehat{SI},\widehat{D'}\rangle$ avec s $\notin \widehat{SI}$ 
					\item $\langle\langle C',C''\rangle $ $\widehat{S},\widehat{\varepsilon},present_{s}$ $\widehat{C},\emptyset,\widehat{ST},\widehat{SI},\widehat{D}\rangle \longmapsto_{secdv1} \langle\emptyset,\emptyset,\emptyset,\emptyset,\widehat{ST}\langle s,\langle\widehat{S},\widehat{\varepsilon},present_{s}$ $\widehat{C},\widehat{D}\rangle\rangle,\emptyset,\emptyset\rangle$ avec s $\notin \widehat{SI}$ 
					\item $\langle\widehat{S},\widehat{\varepsilon},\emptyset,\langle\widehat{S'},\widehat{\varepsilon'},\widehat{C'},\widehat{D'}\rangle$ $\widehat{W},\widehat{ST},\widehat{SI},\emptyset\rangle \longmapsto_{secdv1} \langle\widehat{S'},\widehat{\varepsilon'},\widehat{C'},\widehat{W},\widehat{ST},\widehat{SI},\widehat{D'}\rangle$
					\item  $\langle\widehat{S},\widehat{\varepsilon},\emptyset,\emptyset,\widehat{ST},\widehat{SI},\emptyset\rangle \longmapsto_{secdv1} \langle\widehat{S},\widehat{\varepsilon},\emptyset,\widehat{W},\emptyset,\emptyset,\emptyset\rangle$ avec $\widehat{W}$ = tout les éléments de $\widehat{ST}$ qui prennent leurs 2nd choix 
					\item $\langle\widehat{S},\widehat{\varepsilon},emit_{s}$ $\widehat{C},\widehat{W},\widehat{ST},\widehat{SI},\widehat{D}\rangle \longmapsto_{secdv1} \langle \widehat{S},\widehat{\varepsilon},\widehat{C},\widehat{W'},\widehat{ST'},\widehat{SI},\widehat{D}\rangle$\\ avec $\widehat{W'}$ = $\widehat{W}$ $\cup$ tout les éléments de $\widehat{ST}$ qui attendent l'émission de s et\\
					avec $\widehat{ST'}$ = $\widehat{ST}$ $\setminus$ tout les éléments de $\widehat{ST}$ qui attendent l'émission de s
				\end{enumerate}
				\bigbreak
				
				
				la machine SECD version 1 peut s'arrêter dans 3 états différents:
				\begin{itemize}
					\item[]$\longrightarrow$ on a une \textbf{constante b} tels que $\langle \epsilon,\emptyset,\textlbrackdbl M\textrbrackdbl_{secdv1},\emptyset,\emptyset,\emptyset,\epsilon\rangle \twoheadrightarrow_{secdv1} \langle b,\widehat{\varepsilon},\epsilon,\emptyset,\emptyset,\widehat{SI},\epsilon\rangle$;
					\item[]$\longrightarrow$ on a une \textbf{abstraction function} tels que $\langle \epsilon,\emptyset,\textlbrackdbl M\textrbrackdbl_{secdv1},\emptyset,\emptyset,\emptyset,\epsilon\rangle \twoheadrightarrow_{secdv1} \langle\langle\langle X,\widehat{C}\rangle,\widehat{\varepsilon'}\rangle,\widehat{\varepsilon},\epsilon,\emptyset,\emptyset,\widehat{SI},\epsilon\rangle$;
					\item[]$\longrightarrow$ on a un \textbf{remplacement} tels que $\langle \epsilon,\emptyset,\textlbrackdbl M\textrbrackdbl_{secdv1},\emptyset,\emptyset,\emptyset,\epsilon\rangle \twoheadrightarrow_{secdv1} \langle Remp,\widehat{\varepsilon},\epsilon,\emptyset,\emptyset,\widehat{SI},\epsilon\rangle$;
					\item[]$\longrightarrow$ on a un \textbf{état inconnu} soit une \textbf{erreur}.
				\end{itemize}
				\newpage
				
				
				
				\subsubsection{2ème version des règles de la machine SECD Concurrente}\label{SECDConc2}
				Soit $\langle\widehat{S},\widehat{\varepsilon},\widehat{C},\widehat{W},\widehat{ST},\widehat{SI},\widehat{D}\rangle$ avec :
				
				\begin{itemize}
					\item[] $\widehat{S}$  = ... 
					\begin{itemize}
						\item[|] \sout{$\langle C',C''\rangle$ $\widehat{S}$}
					\end{itemize}
					\item[] $\widehat{\varepsilon}$ = ... 
					\begin{itemize}
						\item[|] s (un signal initialisé)
					\end{itemize}
					\item[] $\widehat{C}$  = ... 
					\begin{itemize}
						\item[|] [ $\langle C',C''\rangle$ $\widehat{C}$ 
						\item[|] $present_{s}$ $\widehat{C}$ ] changé en $\langle s,C',C''\rangle$ $\widehat{C}$
						\item[|] $\langle s,\widehat{C'}\rangle$ $\widehat{C}$ 
					\end{itemize}
					\item[] $\widehat{W}$  = ...
					\item[] $\widehat{ST}$ = ...
					\item[] $\widehat{SI}$ = ...
					\item[] $\widehat{D}$  = ...
				\end{itemize}
				\bigbreak
				
				
				Les nouvelles règles sont les suivantes :
				\smallbreak
				\begin{enumerate}
					\item \sout{$\langle\widehat{S},\widehat{\varepsilon},\langle C',C''\rangle$ $\widehat{C},\widehat{W},\widehat{ST},\widehat{SI},\widehat{D}\rangle \longmapsto_{secdv1} \langle\langle C',C''\rangle$ $\widehat{S},\widehat{\varepsilon},\widehat{C},\widehat{W},\widehat{ST},\widehat{SI},\widehat{D}\rangle$}
					\item \sout{$\langle\langle C',C''\rangle $ $\widehat{S},\widehat{\varepsilon},present_{s}$ $\widehat{C},\widehat{W},\widehat{ST},s$ $\widehat{SI},\widehat{D}\rangle \longmapsto_{secdv1} \langle\widehat{S},\widehat{\varepsilon},C'\widehat{C},\widehat{W},\widehat{ST},s$ $\widehat{SI},\widehat{D}\rangle$}
					\item \sout{$\langle\langle C',C''\rangle $ $\widehat{S},\widehat{\varepsilon},present_{s}$ $\widehat{C},\langle\widehat{S'},\widehat{\varepsilon'},\widehat{C'''},\widehat{D'}\rangle$ $\widehat{W},\widehat{ST},\widehat{SI},\widehat{D}\rangle \longmapsto_{secdv1} \langle\widehat{S'},\widehat{\varepsilon'},\widehat{C'''},\widehat{W},\widehat{ST}\langle s,\langle\widehat{S},\widehat{\varepsilon},present_{s}$ $\widehat{C},\widehat{D}\rangle\rangle,\widehat{SI},\widehat{D'}\rangle$ avec s $\notin \widehat{SI}$ }
					\item \sout{$\langle\langle C',C''\rangle $ $\widehat{S},\widehat{\varepsilon},present_{s}$ $\widehat{C},\emptyset,\widehat{ST},\widehat{SI},\widehat{D}\rangle \longmapsto_{secdv1} \langle\emptyset,\emptyset,\emptyset,\emptyset,\widehat{ST}\langle s,\langle\widehat{S},\widehat{\varepsilon},present_{s}$ $\widehat{C},\widehat{D}\rangle\rangle,\emptyset,\emptyset\rangle$ avec s $\notin \widehat{SI}$}
					\item $\langle\widehat{S},\widehat{\varepsilon},\langle s, C',C''\rangle$ $\widehat{C},\widehat{W},\widehat{ST},\widehat{SI},\widehat{D}\rangle \longmapsto_{secdv2} \langle\widehat{S},\widehat{\varepsilon},C'$ $\widehat{C},\widehat{W},\widehat{ST},\widehat{SI},\widehat{D}\rangle$ si s $\in \widehat{SI}$ et s $\in \widehat{\varepsilon}$ 
					\item $\langle\widehat{S},\widehat{\varepsilon},\langle s, C',C''\rangle$ $\widehat{C},\langle\widehat{S'},\widehat{\varepsilon'},C''',\widehat{D'}\rangle\widehat{W},\widehat{ST},\widehat{SI},\widehat{D}\rangle \longmapsto_{secdv2} \langle\widehat{S'},\widehat{\varepsilon'},\widehat{C'''},\widehat{W},\widehat{ST}$ $\langle\widehat{S},\widehat{\varepsilon},\langle s, C',C''\rangle$ $\widehat{C},\widehat{D}\rangle,\widehat{SI},\widehat{D'}\rangle$ si s $\notin \widehat{SI}$ et s $\in \widehat{\varepsilon}$ 
					\item $\langle\widehat{S},\widehat{\varepsilon},\langle s, C',C''\rangle$ $\widehat{C},\emptyset,\widehat{ST},\widehat{SI},\widehat{D}\rangle \longmapsto_{secdv2} \langle\epsilon,\emptyset,\emptyset,\emptyset,\widehat{ST}$ $\langle\widehat{S},\widehat{\varepsilon},\langle s, C',C''\rangle$ $\widehat{C},\widehat{D}\rangle,\widehat{SI},\emptyset\rangle$ si s $\notin \widehat{SI}$ et s $\in \widehat{\varepsilon}$ 
					\item $\langle\widehat{S},\widehat{\varepsilon},\langle s, C'\rangle$ $\widehat{C},\widehat{W},\widehat{ST},\widehat{SI},\widehat{D}\rangle \longmapsto_{secdv2} \langle\epsilon,\widehat{\varepsilon}[init \leftarrow s],C',\widehat{W},\widehat{ST},\widehat{SI},\langle\widehat{S},\widehat{\varepsilon},\widehat{C},\widehat{D}\rangle\rangle$
				\end{enumerate}
				\bigbreak
				
				
				la machine SECD version 2 peut s'arrêter dans 4 états différents:
				\begin{itemize}
					\item[]$\longrightarrow$ on a une \textbf{constante b} tels que $\langle \epsilon,\emptyset,\textlbrackdbl M\textrbrackdbl_{secdv2},\emptyset,\emptyset,\emptyset,\epsilon\rangle \twoheadrightarrow_{secdv2} \langle b,\widehat{\varepsilon},\epsilon,\emptyset,\emptyset,\widehat{SI},\epsilon\rangle$;
					\item[]$\longrightarrow$ on a une \textbf{abstraction function} tels que $\langle \epsilon,\emptyset,\textlbrackdbl M\textrbrackdbl_{secdv2},\emptyset,\emptyset,\emptyset,\epsilon\rangle \twoheadrightarrow_{secdv2} \langle\langle\langle X,\widehat{C}\rangle,\widehat{\varepsilon'}\rangle,\widehat{\varepsilon},\epsilon,\emptyset,\emptyset,\widehat{SI},\epsilon\rangle$;
					\item[]$\longrightarrow$ on a un \textbf{remplacement} tels que $\langle \epsilon,\emptyset,\textlbrackdbl M\textrbrackdbl_{secdv2},\emptyset,\emptyset,\emptyset,\epsilon\rangle \twoheadrightarrow_{secdv2} \langle Remp,\widehat{\varepsilon},\epsilon,\emptyset,\emptyset,\widehat{SI},\epsilon\rangle$;
					\item[]$\longrightarrow$ on a un \textbf{état inconnu} soit une \textbf{erreur}.
				\end{itemize}
				\newpage
				
				\subsubsection{3ème version des règles de la machine SECD Concurrente}\label{SECDConc3}
					Soit $\langle\widehat{S},\widehat{\varepsilon},\widehat{C},\widehat{W},\widehat{ST},\widehat{SI},\widehat{D},\widehat{H}\rangle$ avec :
				
				\begin{itemize}
					\item[] $\widehat{S}$ = ... 
					\item[] $\widehat{\varepsilon}$ = ... 
					\item[] $\widehat{C}$ = ... |  $throw_{e} $ $\widehat{C}$ | $\langle e,\langle\widehat{C'},\langle X,\widehat{C''}\rangle\rangle\rangle$ $\widehat{C}$ 
					\item[] $\widehat{W}$ = ...
					\item[] $\widehat{ST}$ = ...
					\item[] $\widehat{SI}$ = ...
					\item[] $\widehat{D}$ = ...
					\item[] $\widehat{H}$ = $\epsilon$ | $\langle e,\langle\widehat{S},\widehat{\varepsilon},\widehat{C},\widehat{W},\widehat{ST},\widehat{SI},\widehat{D},\widehat{H}\rangle\rangle$ 
					\item[] $\widehat{V}$ = ... | $erreur_{e}$
				\end{itemize}
				\bigbreak
				
				Les nouvelles règles sont les suivantes :
				\smallbreak
				\begin{enumerate}
					\item $\langle\widehat{S},\widehat{\varepsilon},b$ $\widehat{C},\widehat{W},\widehat{ST},\widehat{SI},\widehat{D},\widehat{H}\rangle \longmapsto_{secdv3} \langle b$ $\widehat{S},\widehat{\varepsilon},\widehat{C},\widehat{W},\widehat{ST},\widehat{SI},\widehat{D},\widehat{H}\rangle$
					\item $\langle\widehat{S},\widehat{\varepsilon},X$ $\widehat{C},\widehat{W},\widehat{ST},\widehat{SI},\widehat{D},\widehat{H}\rangle \longmapsto_{secdv3} \langle \widehat{V}$ $\widehat{S},\widehat{\varepsilon},\widehat{C},\widehat{W},\widehat{ST},\widehat{SI},\widehat{D},\widehat{H}\rangle$ où $\widehat{V} = \varepsilon(X)$
					\item $\langle b_{1}$ $...$ $b_{n}\widehat{S},\widehat{\varepsilon},prim_{o^{n}}$ $\widehat{C},\widehat{W},\widehat{ST},\widehat{SI},\widehat{D},\widehat{H}\rangle \longmapsto_{secdv3} \langle \widehat{V}$ $\widehat{S},\widehat{\varepsilon},\widehat{C},\widehat{W},\widehat{ST},\widehat{SI},\widehat{D},\widehat{H}\rangle$ où $\widehat{V} = \delta(o^{n},b_1,...b_{n})$
					\item $\langle\widehat{S},\widehat{\varepsilon},throw_{e}$ $\widehat{C},\widehat{W},\widehat{ST},\widehat{SI},\widehat{D},\widehat{H}\rangle \longmapsto_{secdv3} \langle erreur_{e}$ $\widehat{S},\widehat{\varepsilon},\widehat{C},\widehat{W},\widehat{ST},\widehat{SI},\widehat{D},\widehat{H}\rangle$
					\item $\langle\widehat{S},\widehat{\varepsilon},\langle X,C'\rangle$ $\widehat{C},\widehat{W},\widehat{ST},\widehat{SI},\widehat{D},\widehat{H}\rangle \longmapsto_{secdv3} \langle\langle\langle X,C'\rangle,\varepsilon\rangle$ $\widehat{S},\widehat{\varepsilon},\widehat{C},\widehat{W},\widehat{ST},\widehat{SI},\widehat{D},\widehat{H}\rangle$
					\item $\langle\widehat{V}$ $\langle\langle X,C'\rangle,\varepsilon'\rangle$ $\widehat{S},\widehat{\varepsilon},ap$ $\widehat{C},\widehat{W},\widehat{ST},\widehat{SI},\widehat{D},\widehat{H}\rangle \longmapsto_{secdv3} \langle\epsilon,\varepsilon'[X \leftarrow \widehat{V}],C',\widehat{W},\widehat{ST},\widehat{SI},\langle\widehat{S},\widehat{\varepsilon},\widehat{C},\widehat{D}\rangle,\widehat{H}\rangle$
					\item $\langle\widehat{V}$ $\widehat{S},\widehat{\varepsilon},\emptyset,\langle\widehat{S'},\widehat{\varepsilon'},\widehat{C'},\widehat{W},\widehat{ST},\widehat{SI},\widehat{D}\rangle,\widehat{H}\rangle \longmapsto_{secdv3} \langle \widehat{V}$ $\widehat{S'},\widehat{\varepsilon'},\widehat{C'},\widehat{W},\widehat{ST},\widehat{SI},\widehat{D},\widehat{H}\rangle$
					\item $\langle\widehat{V}$ Remp $\widehat{S},\widehat{\varepsilon},ap$ $\widehat{C},\widehat{W},\widehat{ST},\widehat{SI},\widehat{D},\widehat{H}\rangle \longmapsto_{secdv3} \langle \widehat{V}$ $\widehat{S},\widehat{\varepsilon},\widehat{C},\widehat{W},\widehat{ST},\widehat{SI},\widehat{D},\widehat{H}\rangle$
					\item $\langle$ Remp $\widehat{V}\widehat{S},\widehat{\varepsilon},ap$ $\widehat{C},\widehat{W},\widehat{ST},\widehat{SI},\widehat{D},\widehat{H}\rangle \longmapsto_{secdv3} \langle \widehat{V}$ $\widehat{S},\widehat{\varepsilon},\widehat{C},\widehat{W},\widehat{ST},\widehat{SI},\widehat{D},\widehat{H}\rangle$
					\item $\langle$ Remp $\widehat{V}\widehat{S},\widehat{\varepsilon},ap$ $\widehat{C},\widehat{W},\widehat{ST},\widehat{SI},\widehat{D},\widehat{H}\rangle \longmapsto_{secdv3} \langle \widehat{V}$ $\widehat{S},\widehat{\varepsilon},\widehat{C},\widehat{W},\widehat{ST},\widehat{SI},\widehat{D},\widehat{H}\rangle$
					\item $\langle\widehat{S},\widehat{\varepsilon},bspawn$ $\widehat{C'}$ $espawn$ $\widehat{C},\widehat{W},\widehat{ST},\widehat{SI},\widehat{D},\widehat{H}\rangle \longmapsto_{secdv3} \langle$ Remp $\widehat{S},\widehat{\varepsilon},\widehat{C},\widehat{W}\langle\widehat{S},\widehat{E},\widehat{C'},\widehat{D}\rangle,\widehat{ST},\widehat{SI},\widehat{D},\widehat{H}\rangle$
			    	\item $\langle\widehat{S},\widehat{\varepsilon},\langle s, C',C''\rangle$ $\widehat{C},\widehat{W},\widehat{ST},\widehat{SI},\widehat{D},\widehat{H}\rangle \longmapsto_{secdv3} \langle\widehat{S},\widehat{\varepsilon},C'$ $\widehat{C},\widehat{W},\widehat{ST},\widehat{SI},\widehat{D},\widehat{H}\rangle$ si s $\in \widehat{SI}$ et s $\in \widehat{\varepsilon}$ 
			    	\item $\langle\widehat{S},\widehat{\varepsilon},\langle s, C',C''\rangle$ $\widehat{C},\langle\widehat{S'},\widehat{\varepsilon'},C''',\widehat{D'}\rangle\widehat{W},\widehat{ST},\widehat{SI},\widehat{D},\widehat{H}\rangle \longmapsto_{secdv3} \langle\widehat{S'},\widehat{\varepsilon'},\widehat{C'''},\widehat{W},\widehat{ST}$ $\langle\widehat{S},\widehat{\varepsilon},\langle s, C',C''\rangle$ $\widehat{C},\widehat{D}\rangle,\widehat{SI},\widehat{D'},\widehat{H}\rangle$ si s $\notin \widehat{SI}$ et s $\in \widehat{\varepsilon}$ 
			    	\item $\langle\widehat{S},\widehat{\varepsilon},\langle s, C',C''\rangle$ $\widehat{C},\emptyset,\widehat{ST},\widehat{SI},\widehat{D},\widehat{H}\rangle \longmapsto_{secdv3} \langle\epsilon,\emptyset,\emptyset,\emptyset,\widehat{ST}$ $\langle\widehat{S},\widehat{\varepsilon},\langle s, C',C''\rangle$ $\widehat{C},\widehat{D}\rangle,\widehat{SI},\emptyset,\widehat{H}\rangle$ si s $\notin \widehat{SI}$ et s $\in \widehat{\varepsilon}$ 
				    \item $\langle\widehat{S},\widehat{\varepsilon},\langle s, C'\rangle$ $\widehat{C},\widehat{W},\widehat{ST},\widehat{SI},\widehat{D},\widehat{H}\rangle \longmapsto_{secdv3} \langle\epsilon,\widehat{\varepsilon}[init \leftarrow s],C',\widehat{W},\widehat{ST},\widehat{SI},\langle\widehat{S},\widehat{\varepsilon},\widehat{C},\widehat{D}\rangle,\widehat{H}\rangle$
					\item $\langle\widehat{S},\widehat{\varepsilon},\emptyset,\langle\widehat{S'},\widehat{\varepsilon'},\widehat{C'},\widehat{D'}\rangle$ $\widehat{W},\widehat{ST},\widehat{SI},\emptyset,\widehat{H}\rangle \longmapsto_{secdv3} \langle\widehat{S'},\widehat{\varepsilon'},\widehat{C'},\widehat{W},\widehat{ST},\widehat{SI},\widehat{D'},\widehat{H}\rangle$
					\item  $\langle\widehat{S},\widehat{\varepsilon},\emptyset,\emptyset,\widehat{ST},\widehat{SI},\emptyset,\widehat{H}\rangle \longmapsto_{secdv3} \langle\widehat{S},\widehat{\varepsilon},\emptyset,\widehat{W},\emptyset,\emptyset,\emptyset,\widehat{H}\rangle$\\ avec $\widehat{W}$ = tout les éléments de $\widehat{ST}$ qui prennent leurs 2nd choix 
					\item $\langle\widehat{S},\widehat{\varepsilon},emit_{s}$ $\widehat{C},\widehat{W},\widehat{ST},\widehat{SI},\widehat{D},\widehat{H}\rangle \longmapsto_{secdv3} \langle \widehat{S},\widehat{\varepsilon},\widehat{C},\widehat{W'},\widehat{ST'},\widehat{SI},\widehat{D},\widehat{H}\rangle$\\ avec $\widehat{W'}$ = $\widehat{W}$ $\cup$ tout les éléments de $\widehat{ST}$ qui attendent l'émission de s et\\
					avec $\widehat{ST'}$ = $\widehat{ST}$ $\setminus$ tout les éléments de $\widehat{ST}$ qui attendent l'émission de s
					\item $\langle\widehat{S},\widehat{\varepsilon},\langle e,\langle\widehat{C'},\langle X,\widehat{C''}\rangle\rangle\rangle$ $\widehat{C},\widehat{W},\widehat{ST},\widehat{SI},\widehat{D},\widehat{H}\rangle \longmapsto_{secdv3} \langle\widehat{S},\widehat{\varepsilon},\widehat{C'}$ $\widehat{C},\widehat{W},\widehat{ST},\widehat{SI},\widehat{D},\langle e,\langle\widehat{S},\widehat{\varepsilon},\langle X,\widehat{C''}\rangle$ $\widehat{C},\widehat{W},\widehat{ST},\widehat{SI},\widehat{D},\widehat{H}\rangle\rangle\rangle$
					\item $\langle\widehat{S},\widehat{\varepsilon},throw_{e}$ $\widehat{C},\widehat{W},\widehat{ST},\widehat{SI},\widehat{D},\langle e,\langle\widehat{S'},\widehat{\varepsilon'},\langle X,\widehat{C''}\rangle$ $\widehat{C'},\widehat{W'},\widehat{ST'},\widehat{SI'},\widehat{D'},\widehat{H}\rangle\rangle\rangle \longmapsto_{secdv3}$\\$ \langle\widehat{S'},\widehat{\varepsilon'},\langle X,\widehat{C''}\rangle$ $throw_{e}$ $\widehat{C},\widehat{W'},\widehat{ST'},\widehat{SI'},\widehat{D'},\widehat{H}\rangle$
					\item $\langle\widehat{S},\widehat{\varepsilon},throw_{e}$ $\widehat{C},\widehat{W},\widehat{ST},\widehat{SI},\widehat{D},\langle e',\langle\widehat{S'},\widehat{\varepsilon'},\langle X,\widehat{C''}\rangle$ $\widehat{C'},\widehat{W'},\widehat{ST'},\widehat{SI'},\widehat{D'},\widehat{H}\rangle\rangle\rangle \longmapsto_{secdv3}$\\on regarde dans $\widehat{H}$ récursivement si il y a un gestionnaire pour l'erreur e
					\item $\langle\widehat{S},\widehat{\varepsilon},throw_{e}$ $\widehat{C},\widehat{W},\widehat{ST},\widehat{SI},\widehat{D},\emptyset\rangle \longmapsto_{secdv3} \langle\widehat{S},\widehat{\varepsilon},throw_{e},\emptyset,\emptyset,\emptyset,\emptyset,\emptyset\rangle$  
				\end{enumerate}
				\bigbreak
				
				
				la machine SECD version 3 peut s'arrêter dans 4 états différents:
				\begin{itemize}
					\item[]$\longrightarrow$ on a une \textbf{constante b} tels que $\langle \epsilon,\emptyset,\textlbrackdbl M\textrbrackdbl_{secdv3},\emptyset,\emptyset,\emptyset,\epsilon,\emptyset\rangle \twoheadrightarrow_{secdv3} \langle b,\widehat{\varepsilon},\epsilon,\emptyset,\emptyset,\widehat{SI},\epsilon,\widehat{H}\rangle$;
					\item[]$\longrightarrow$ on a une \textbf{abstraction function} tels que $\langle \epsilon,\emptyset,\textlbrackdbl M\textrbrackdbl_{secdv3},\emptyset,\emptyset,\emptyset,\epsilon,\emptyset\rangle \twoheadrightarrow_{secdv3} \langle\langle\langle X,\widehat{C}\rangle,\widehat{\varepsilon'}\rangle,\widehat{\varepsilon},\epsilon,\emptyset,\emptyset,\widehat{SI},\epsilon,\widehat{H}\rangle$;
					\item[]$\longrightarrow$ on a un \textbf{remplacement} tels que $\langle \epsilon,\emptyset,\textlbrackdbl M\textrbrackdbl_{secdv3},\emptyset,\emptyset,\emptyset,\epsilon,\emptyset\rangle \twoheadrightarrow_{secdv3} \langle Remp,\widehat{\varepsilon},\epsilon,\emptyset,\emptyset,\widehat{SI},\epsilon,\widehat{H}\rangle$;
					\item[]$\longrightarrow$ on a une \textbf{erreur e} tels que $\langle \epsilon,\emptyset,\textlbrackdbl M\textrbrackdbl_{secdv3},\emptyset,\emptyset,\emptyset,\epsilon,\emptyset\rangle \twoheadrightarrow_{secdv3} \langle erreur_{e},\widehat{\varepsilon},\epsilon,\emptyset,\emptyset,\widehat{SI},\epsilon,\widehat{H}\rangle$;
				\end{itemize}
		\newpage
		\section{Bibliographie}
			\begin{itemize}
				\item[] [1] \textit{Réactivité des systèmes coopératifs : le cas Réactive ML} de Louis Mandrel et Cédric Pasteur\label{ReactiveML}
				\item[] [2] \textit{The ZINC experiment: an economical implementation of the ML language} de Xavier Leroy\label{ZINC}
				\item[] [3] \textit{Programming Languages And Lambda Calculi} de Mathias Felleisen et Matthew Flatt\label{Calculi}
				\item[] [4] \textit{https://www.irif.fr/~carton/Enseignement/Complexite/MasterInfo/Cours/turing.html}\label{Turing}
				\item[] [5] \textit{https://fr.wikipedia.org/wiki/Automate$\_$fini$\_$déterministe}\label{AFD}
				\item[] [6] \textit{https://fr.wikipedia.org/wiki/Algorithme$\_$déterministe}\label{MFD}
			\end{itemize}
\end{document}